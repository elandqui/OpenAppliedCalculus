\protect\hyperlink{main-nav}{≡} \protect\hyperlink{close-nav}{×}

\hypertarget{section-3.4-substitution}{%
\section{Section 3.4: Substitution}\label{section-3.4-substitution}}

We don't have many integration rules. For quite a few of the problems we
see, the rules won't directly apply; we'll have to do some algebraic
manipulation first. In practice, it is much harder to write down the
antiderivative of a function than it is to find a derivative. (In fact,
it's very easy to write a function that doesn't have any antiderivative
you can find with algebra, although \emph{proving} that it doesn't have
an antiderivative is much more difficult.)

The Substitution Method (also called \textbackslash{}( u
\textbackslash{})-Substitution) is one way of algebraically manipulating
an integrand so that the rules apply. This is a way to unwind or undo
the Chain Rule for derivatives. When you find the derivative of a
function using the Chain Rule, you end up with a product of something
like the original function \emph{times} a derivative. We can reverse
this to write an integral: \textbackslash{}{[}
\textbackslash{}frac\{d\}\{dx\} f\textbackslash{}left( g(x)
\textbackslash{}right) = f'\textbackslash{}left( g(x)
\textbackslash{}right)g'(x) \textbackslash{}{]} so \textbackslash{}{[}
f\textbackslash{}left( g(x) \textbackslash{}right) =\textbackslash{}int
f'\textbackslash{}left( g(x)
\textbackslash{}right)g'(x)\textbackslash{}, dx\textbackslash{}{]}

With substitution, we will substitute \textbackslash{}( u=g(x)
\textbackslash{}) (hence the name ``\textbackslash{}( u
\textbackslash{})-substitution''). This means \textbackslash{}(
\textbackslash{}frac\{du\}\{dx\}=g'(x) \textbackslash{}), so
\textbackslash{}( du=g'(x)dx \textbackslash{}). Making these
substitutions, \textbackslash{}( \textbackslash{}int
f'\textbackslash{}left( g(x)
\textbackslash{}right)g'(x)\textbackslash{}, dx \textbackslash{})
becomes \textbackslash{}( \textbackslash{}int f'(u)\textbackslash{}, du
\textbackslash{}), which will probably be easier to integrate.

Try \textbackslash{}( u \textbackslash{})-Substitution when you see a
product in your integral, especially if you recognize one factor as the
derivative of some part of the other factor.

To view this video please enable JavaScript, and consider upgrading to a
web browser that \href{http://videojs.com/html5-video-support/}{supports
HTML5 video}

\hypertarget{the-u--substitution-method-for-antiderivatives}{%
\paragraph{The \textbackslash{}( u \textbackslash{})-Substitution Method
for
Antiderivatives}\label{the-u--substitution-method-for-antiderivatives}}

The goal is to turn \textbackslash{}( \textbackslash{}int
f\textbackslash{}left( g(x) \textbackslash{}right)\textbackslash{}, dx
\textbackslash{}) into \textbackslash{}( \textbackslash{}int
f(u)\textbackslash{}, du \textbackslash{}), where
\textbackslash{}(f(u)\textbackslash{}) is much less messy than
\textbackslash{}(f\textbackslash{}left(g(x)\textbackslash{}right)\textbackslash{}).

\begin{enumerate}
\tightlist
\item
  Let \textbackslash{}(u\textbackslash{}) be some part of the integrand.
  A good first choice is ``one step inside the messiest bit.''
\item
  Compute \textbackslash{}( du=\textbackslash{}frac\{du\}\{dx\}dx
  \textbackslash{}).
\item
  Translate all your \textbackslash{}(x\textbackslash{})'s into
  \textbackslash{}(u\textbackslash{})'s everywhere in the integral,
  including the \textbackslash{}(dx\textbackslash{}). When you're done,
  you should have a new integral that is entirely in
  \textbackslash{}(u\textbackslash{}). If you have any
  \textbackslash{}(x\textbackslash{})'s left, then that's an indication
  that the substitution didn't work or isn't complete; you may need to
  go back to step 1 and try a different choice for
  \textbackslash{}(u\textbackslash{}).
\item
  Integrate the new \textbackslash{}(u\textbackslash{})-integral, if
  possible. If you still can't integrate it, go back to step 1 and try a
  different choice for \textbackslash{}(u\textbackslash{}).
\item
  Finally, substitute back \textbackslash{}(x\textbackslash{})'s for
  \textbackslash{}(u\textbackslash{})'s everywhere in your answer.
\end{enumerate}

\hypertarget{example-1}{%
\paragraph{Example 1}\label{example-1}}

Evaluate \textbackslash{}( \textbackslash{}int
\textbackslash{}frac\{x\}\{\textbackslash{}sqrt\{4-x\^{}2\}\}\textbackslash{},
dx \textbackslash{}).

This integrand is more complicated than anything in our list of basic
integral formulas, so we'll have to try something else. The only tool we
have is substitution, so let's try that!

\begin{enumerate}
\item
  Let \textbackslash{}(u\textbackslash{}) be some part of the integrand.
  A good first choice is ``one step inside the messiest bit'':

  Let \textbackslash{}( u=4-x\^{}2 \textbackslash{}).
\item
  Compute \textbackslash{}( du=\textbackslash{}frac\{du\}\{dx\}dx
  \textbackslash{}):

  \textbackslash{}( du=-2x\textbackslash{}, dx \textbackslash{}). There
  is \textbackslash{}(x\textbackslash{}, dx\textbackslash{}) in the
  integrand, so that's a good sign; that will be
  \textbackslash{}(-\textbackslash{}frac\{1\}\{2\}du\textbackslash{}).
\item
  Translate all your \textbackslash{}(x\textbackslash{})'s into
  \textbackslash{}(u\textbackslash{})'s everywhere in the integral,
  including the \textbackslash{}(dx\textbackslash{}):

  \textbackslash{}{[} \textbackslash{}begin\{align*\}
  \textbackslash{}int\textbackslash{}frac\{x\}\{\textbackslash{}sqrt\{4-x\^{}2\}\}\textbackslash{},
  dx=\&
  \textbackslash{}int\textbackslash{}frac\{1\}\{\textbackslash{}sqrt\{4-x\^{}2\}\}(x\textbackslash{},
  dx) \textbackslash{}\textbackslash{} =\&
  \textbackslash{}int\textbackslash{}frac\{1\}\{\textbackslash{}sqrt\{u\}\}\textbackslash{}left(-\textbackslash{}frac\{1\}\{2\}du\textbackslash{}right)
  \textbackslash{}\textbackslash{} =\&
  -\textbackslash{}frac\{1\}\{2\}\textbackslash{}int\textbackslash{}frac\{1\}\{\textbackslash{}sqrt\{u\}\}\textbackslash{},
  du \textbackslash{}\textbackslash{} =\&
  -\textbackslash{}frac\{1\}\{2\}\textbackslash{}int
  u\^{}\{-1/2\}\textbackslash{}, du \textbackslash{}end\{align*\}
  \textbackslash{}{]}

  Alternatively, we could have solved for dx and substituted that and
  simplified: \textbackslash{}( dx=\textbackslash{}frac\{du\}\{-2x\}
  \textbackslash{}), so \textbackslash{}{[}
  \textbackslash{}begin\{align*\}
  \textbackslash{}int\textbackslash{}frac\{x\}\{\textbackslash{}sqrt\{4-x\^{}2\}\}\textbackslash{},
  dx=\&
  \textbackslash{}int\textbackslash{}frac\{x\}\{\textbackslash{}sqrt\{u\}\}\textbackslash{}left(\textbackslash{}frac\{du\}\{-2x\}\textbackslash{}right)
  \textbackslash{}\textbackslash{} =\&
  \textbackslash{}int\textbackslash{}frac\{1\}\{\textbackslash{}sqrt\{u\}\}\textbackslash{}left(-\textbackslash{}frac\{1\}\{2\}du\textbackslash{}right)
  \textbackslash{}\textbackslash{} =\&
  -\textbackslash{}frac\{1\}\{2\}\textbackslash{}int\textbackslash{}frac\{1\}\{\textbackslash{}sqrt\{u\}\}\textbackslash{},
  du \textbackslash{}\textbackslash{} =\&
  -\textbackslash{}frac\{1\}\{2\}\textbackslash{}int
  u\^{}\{-1/2\}\textbackslash{}, du \textbackslash{}end\{align*\}
  \textbackslash{}{]}
\item
  Integrate the new \textbackslash{}(u\textbackslash{})-integral, if
  possible:

  \textbackslash{}{[} -\textbackslash{}frac\{1\}\{2\}\textbackslash{}int
  u\^{}\{-1/2\}\textbackslash{}, du =
  -\textbackslash{}frac\{1\}\{2\}\textbackslash{}frac\{u\^{}\{1/2\}\}\{1/2\}+C=-u\^{}\{1/2\}+C
  \textbackslash{}{]}
\item
  Finally, substitute back \textbackslash{}(x\textbackslash{})'s for
  \textbackslash{}(u\textbackslash{})'s everywhere in the answer:

  Undoing our \textbackslash{}( u=4-x\^{}2 \textbackslash{})
  substitution yields \textbackslash{}{[} -u\^{}\{1/2\}+C =
  -\textbackslash{}sqrt\{4-x\^{}2\}+C. \textbackslash{}{]}
\end{enumerate}

Thus we have found \textbackslash{}{[}
\textbackslash{}int\textbackslash{}frac\{x\}\{\textbackslash{}sqrt\{4-x\^{}2\}\}\textbackslash{},
dx= -\textbackslash{}sqrt\{4-x\^{}2\}+C \textbackslash{}{]}

How would we check this? By differentiating: \textbackslash{}{[}
\textbackslash{}begin\{align*\}
\textbackslash{}frac\{d\}\{dx\}\textbackslash{}left(-\textbackslash{}sqrt\{4-x\^{}2\}+C\textbackslash{}right)=\&
\textbackslash{}frac\{d\}\{dx\}\textbackslash{}left(-\textbackslash{}left(4-x\^{}2\textbackslash{}right)\^{}\{1/2\}+C\textbackslash{}right)
\textbackslash{}\textbackslash{} =\&
-\textbackslash{}frac\{1\}\{2\}\textbackslash{}left(4-x\^{}2\textbackslash{}right)\^{}\{-1/2\}(-2x)
\textbackslash{}\textbackslash{} =\&
x\textbackslash{}left(4-x\^{}2\textbackslash{}right)\^{}\{-1/2\}
\textbackslash{}\textbackslash{} =\&
\textbackslash{}frac\{x\}\{\textbackslash{}sqrt\{4-x\^{}2\}\}
\textbackslash{}end\{align*\} \textbackslash{}{]}

To view this video please enable JavaScript, and consider upgrading to a
web browser that \href{http://videojs.com/html5-video-support/}{supports
HTML5 video}

\hypertarget{example-2}{%
\paragraph{Example 2}\label{example-2}}

Evaluate \textbackslash{}(
\textbackslash{}int\textbackslash{}frac\{e\^{}x\textbackslash{},
dx\}\{\textbackslash{}left(e\^{}x+15\textbackslash{}right)\^{}3\}
\textbackslash{})

This integral is not in our list of building blocks. But notice that the
derivative of \textbackslash{}( e\^{}x+15 \textbackslash{}) (which we
see in the denominator) is just \textbackslash{}( e\^{}x
\textbackslash{}) (which we see in the numerator), so substitution will
be a good choice for this.

Let \textbackslash{}( u=e\^{}x+15 \textbackslash{}). Then
\textbackslash{}( du=e\^{}x\textbackslash{}, dx \textbackslash{}), and
this integral becomes \textbackslash{}(
\textbackslash{}int\textbackslash{}frac\{du\}\{u\^{}3\} =
\textbackslash{}int u\^{}\{-3\}\textbackslash{}, du \textbackslash{}).

Luckily, that is on our list of building block formulas:
\textbackslash{}(
\textbackslash{}int\textbackslash{}frac\{du\}\{u\^{}3\} =
\textbackslash{}frac\{u\^{}\{-2\}\}\{-2\}+C =
-\textbackslash{}frac\{1\}\{2u\^{}2\}+C \textbackslash{}).

Finally, translating back: \textbackslash{}{[}
\textbackslash{}int\textbackslash{}frac\{e\^{}x\textbackslash{},
dx\}\{\textbackslash{}left(e\^{}x+15\textbackslash{}right)\^{}3\} =
-\textbackslash{}frac\{1\}\{2\textbackslash{}left(e\^{}x+15\textbackslash{}right)\^{}2\}
+C. \textbackslash{}{]}

\hypertarget{example-3}{%
\paragraph{Example 3}\label{example-3}}

Evaluate

\begin{enumerate}
\tightlist
\item
  \textbackslash{}(
  \textbackslash{}int\textbackslash{}frac\{x\^{}2\}\{x\^{}3+5\}\textbackslash{},
  dx \textbackslash{})
\item
  \textbackslash{}(
  \textbackslash{}int\textbackslash{}frac\{x\^{}3+5\}\{x\^{}2\}\textbackslash{},
  dx \textbackslash{})
\end{enumerate}

\begin{enumerate}
\item
  This is not a basic integral, but the composition is less obvious.
  Here, we can treat the denominator as the inside of the
  \textbackslash{}( \textbackslash{}frac\{1\}\{x\} \textbackslash{})
  function.

  Let \textbackslash{}( u=x\^{}3+5 \textbackslash{}). Then
  \textbackslash{}( du=3x\^{}2\textbackslash{}, dx \textbackslash{}).
  Solving for \textbackslash{}(dx\textbackslash{}), \textbackslash{}(
  dx=\textbackslash{}frac\{du\}\{3x\^{}2\} \textbackslash{}).
  Substituting, \textbackslash{}{[}
  \textbackslash{}int\textbackslash{}frac\{x\^{}2\}\{x\^{}3+5\}\textbackslash{},
  dx =
  \textbackslash{}int\textbackslash{}frac\{x\^{}2\}\{u\}\textbackslash{}frac\{du\}\{3x\^{}2\}
  = \textbackslash{}int
  \textbackslash{}frac\{1\}\{u\}\textbackslash{}frac\{du\}\{3\} =
  \textbackslash{}frac\{1\}\{3\}\textbackslash{}int
  \textbackslash{}frac\{1\}\{u\}\textbackslash{}, du \textbackslash{}{]}

  Using our basic formulas, \textbackslash{}{[}
  \textbackslash{}frac\{1\}\{3\}\textbackslash{}int
  \textbackslash{}frac\{1\}\{u\}\textbackslash{}, du =
  \textbackslash{}frac\{1\}\{3\}\textbackslash{}ln\textbar{}u\textbar{}
  +C. \textbackslash{}{]}

  Undoing the substitution, \textbackslash{}{[}
  \textbackslash{}int\textbackslash{}frac\{x\^{}2\}\{x\^{}3+5\}\textbackslash{},
  dx =
  \textbackslash{}frac\{1\}\{3\}\textbackslash{}ln\textbackslash{}left\textbar{}x\^{}3+5\textbackslash{}right\textbar{}
  +C. \textbackslash{}{]}
\item
  It is tempting to start this problem the same way we did the last, but
  if we try it will not work, since the numerator of this fraction is
  not the derivative of the denominator. Instead, we need to try a
  different approach. For this problem, we can use some basic algebra:
  \textbackslash{}{[} \textbackslash{}begin\{align*\}
  \textbackslash{}int\textbackslash{}frac\{x\^{}3+5\}\{x\^{}2\}\textbackslash{},
  dx =\&
  \textbackslash{}int\textbackslash{}left(\textbackslash{}frac\{x\^{}3\}\{x\^{}2\}+\textbackslash{}frac\{5\}\{x\^{}2\}\textbackslash{}right)\textbackslash{},
  dx \textbackslash{}\textbackslash{} =\&
  \textbackslash{}int\textbackslash{}left(x+5x\^{}\{-2\}\textbackslash{}right)\textbackslash{},
  dx. \textbackslash{}end\{align*\} \textbackslash{}{]}

  We can integrate this using our basic rules, without needing
  substitution: \textbackslash{}{[} \textbackslash{}begin\{align*\}
  \textbackslash{}int\textbackslash{}left(x+5x\^{}\{-2\}\textbackslash{}right)\textbackslash{},
  dx=\&
  \textbackslash{}frac\{x\^{}2\}\{2\}+5\textbackslash{}frac\{x\^{}\{-1\}\}\{-1\}+C
  \textbackslash{}\textbackslash{} =\&
  \textbackslash{}frac\{1\}\{2\}x\^{}2-\textbackslash{}frac\{5\}\{x\}+C.
  \textbackslash{}end\{align*\} \textbackslash{}{]}
\end{enumerate}

\hypertarget{substitution-and-definite-integrals}{%
\subsection{Substitution and Definite
Integrals}\label{substitution-and-definite-integrals}}

When you use substitution to help evaluate a definite integral, you have
a choice for how to handle the limits of integration. You can do either
of these, whichever seems better to you. The important thing to remember
is that the original limits of integration were values of the original
variable (say, \textbackslash{}(x\textbackslash{})), not values of the
new variable (say, \textbackslash{}(u\textbackslash{})).

\begin{enumerate}
\item
  You can solve the antiderivative as a side problem, translating back
  to \textbackslash{}(x\textbackslash{})'s, and then use the
  antiderivative with the original limits of integration.

  Or\ldots{}
\item
  You can substitute for the limits of integration at the same time as
  you're substituting for everything inside the integral, and then skip
  the ``translate back into \textbackslash{}(x\textbackslash{})'' step.

  If the original integral had endpoints \textbackslash{}(x
  =a\textbackslash{}) and \textbackslash{}(x =b\textbackslash{}), and we
  make the substitution \textbackslash{}(u = g(x )\textbackslash{}) and
  \textbackslash{}(du = g'(x )\textbackslash{}, dx\textbackslash{}),
  then the new integral will have endpoints \textbackslash{}(u=
  g(a)\textbackslash{}) and \textbackslash{}(u=g(b)\textbackslash{}) and
  \textbackslash{}{[}
  \textbackslash{}int\_\{x=a\}\^{}\{x=b\}\textbackslash{}text\{(original
  integrand)\}\textbackslash{}, dx\textbackslash{}{]} becomes
  \textbackslash{}{[} \textbackslash{}int\_\{u=g(a)\}\^{}\{u=g(b)\}
  \textbackslash{}text\{(new integrand)\}\textbackslash{},
  du.\textbackslash{}{]}
\end{enumerate}

Method 1 seems more straightforward for most students, but it can
involve some messy algebra. Method 2 is often neater and usually
involves fewer steps.

\hypertarget{example-4}{%
\paragraph{Example 4}\label{example-4}}

Evaluate \textbackslash{}(
\textbackslash{}int\textbackslash{}limits\_0\^{}1
(3x-1)\^{}4\textbackslash{}, dx \textbackslash{}).

We'll need substitution to find an antiderivative, so we'll need to
handle the limits of integration carefully. Let's solve this example
both ways.

\begin{enumerate}
\item
  \textbf{Doing the antiderivative as a side problem:}

  Step One -- find the antiderivative, using substitution:

  Let \textbackslash{}( u=3x-1 \textbackslash{}). Then \textbackslash{}(
  du=3\textbackslash{}, dx \textbackslash{}) and \textbackslash{}{[}
  \textbackslash{}int(3x-1)\^{}4\textbackslash{}, dx =
  \textbackslash{}int
  u\^{}4\textbackslash{}left(\textbackslash{}frac\{1\}\{3\}\textbackslash{},
  du\textbackslash{}right) =
  \textbackslash{}frac\{1\}\{3\}\textbackslash{}frac\{u\^{}5\}\{5\}+C.
  \textbackslash{}{]}

  Translating back to x: \textbackslash{}{[}
  \textbackslash{}frac\{1\}\{3\}\textbackslash{}frac\{u\^{}5\}\{5\}+C =
  \textbackslash{}frac\{(3x-1)\^{}5\}\{15\}+C. \textbackslash{}{]}

  Step Two -- evaluate the definite integral: \textbackslash{}{[}
  \textbackslash{}int\textbackslash{}limits\_0\^{}1
  (3x-1)\^{}4\textbackslash{}, dx = \textbackslash{}left.
  \textbackslash{}frac\{(3x-1)\^{}5\}\{15\}\textbackslash{}right{]}\_0\^{}1
  =
  \textbackslash{}frac\{\textbackslash{}left(3(1)-1\textbackslash{}right)\^{}5\}\{15\}
  -
  \textbackslash{}frac\{\textbackslash{}left(3(0)-1\textbackslash{}right)\^{}5\}\{15\}
  =
  \textbackslash{}frac\{32\}\{15\}-\textbackslash{}frac\{-1\}\{15\}=\textbackslash{}frac\{33\}\{15\}.
  \textbackslash{}{]}
\item
  \textbf{Substituting for the limits of integration:}

  Let \textbackslash{}( u=3x-1 \textbackslash{}). Then \textbackslash{}(
  du=3\textbackslash{}, dx \textbackslash{}) and, substituting for the
  limits of integration, when \textbackslash{}(x = 0\textbackslash{}),
  \textbackslash{}(u = -1\textbackslash{}), and when \textbackslash{}(x
  = 1\textbackslash{}), \textbackslash{}(u = 2\textbackslash{}).

  So, \textbackslash{}{[} \textbackslash{}begin\{align*\}
  \textbackslash{}int\_\{x=0\}\^{}\{x=1\} (3x-1)\^{}4\textbackslash{},
  dx =\& \textbackslash{}int\_\{u=-1\}\^{}\{u=2\}
  u\^{}4\textbackslash{}left(\textbackslash{}frac\{1\}\{3\}\textbackslash{},
  du\textbackslash{}right) \textbackslash{}\textbackslash{} =\&
  \textbackslash{}left.\textbackslash{}frac\{u\^{}5\}\{15\}\textbackslash{}right{]}\_\{u=-1\}\^{}\{u=2\}
  \textbackslash{}\textbackslash{} =\&
  \textbackslash{}frac\{(2)\^{}5\}\{15\}-\textbackslash{}frac\{(-1)\^{}5\}\{15\}
  \textbackslash{}\textbackslash{} =\&
  \textbackslash{}frac\{32\}\{15\}-\textbackslash{}frac\{-1\}\{15\}
  \textbackslash{}\textbackslash{} =\& \textbackslash{}frac\{33\}\{15\}
  \textbackslash{}end\{align*\} \textbackslash{}{]}
\end{enumerate}

To view this video please enable JavaScript, and consider upgrading to a
web browser that \href{http://videojs.com/html5-video-support/}{supports
HTML5 video}

\hypertarget{example-5}{%
\paragraph{Example 5}\label{example-5}}

Evaluate \textbackslash{}(
\textbackslash{}int\textbackslash{}limits\_2\^{}\{10\}
\textbackslash{}frac\{\textbackslash{}left(\textbackslash{}ln(x)\textbackslash{}right)\^{}6\}\{x\}\textbackslash{},
dx \textbackslash{}).

I can see the derivative of \textbackslash{}( \textbackslash{}ln(x)
\textbackslash{}) in the integrand, so I can tell that substitution is a
good choice.

Let \textbackslash{}( u=\textbackslash{}ln(x) \textbackslash{}). Then
\textbackslash{}( du=\textbackslash{}frac\{1\}\{x\}\textbackslash{}, dx
\textbackslash{}). When \textbackslash{}( x=2 \textbackslash{}),
\textbackslash{}( u=\textbackslash{}ln(2) \textbackslash{}). When
\textbackslash{}( x=10 \textbackslash{}), \textbackslash{}(
u=\textbackslash{}ln(10) \textbackslash{}). So the new definite integral
is \textbackslash{}{[} \textbackslash{}begin\{align*\}
\textbackslash{}int\textbackslash{}limits\_\{x=2\}\^{}\{x=10\}\textbackslash{}frac\{\textbackslash{}left(\textbackslash{}ln(x)\textbackslash{}right)\^{}6\}\{x\}\textbackslash{},
dx =\&
\textbackslash{}int\textbackslash{}limits\_\{u=\textbackslash{}ln(2)\}\^{}\{u=\textbackslash{}ln(10)\}
u\^{}6\textbackslash{}, du \textbackslash{}\textbackslash{} =\&
\textbackslash{}left.\textbackslash{}frac\{u\^{}7\}\{7\}\textbackslash{}right{]}\_\{u=\textbackslash{}ln(2)\}\^{}\{u=\textbackslash{}ln(10)\}
\textbackslash{}\textbackslash{} =\&
\textbackslash{}frac\{1\}\{7\}\textbackslash{}left(\textbackslash{}left(\textbackslash{}ln(10)\textbackslash{}right)\^{}7-\textbackslash{}left(\textbackslash{}ln(2)\textbackslash{}right)\^{}7\textbackslash{}right)
\textbackslash{}\textbackslash{} \textbackslash{}approx \& 49.01.
\textbackslash{}end\{align*\} \textbackslash{}{]}

\begin{longtable}[]{@{}ll@{}}
\toprule
\endhead
\href{section3-3.php}{← Previous Section} & \href{section3-5.php}{Next
Section →}\tabularnewline
\bottomrule
\end{longtable}
