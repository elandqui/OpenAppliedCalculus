\section{Higher Order Derivatives}
\label{sec:higherorder}

\subsection{Second Derivative}
\label{ssec:second-deriv}

\begin{definition}[Second Derivative\index{Second derivative}\index{Derivative!second}]
Let $y=f(x)$. The {\bf second derivative} of $f(x)$ is the derivative of its derivative: $\dfrac{d}{dx}y'=\dfrac{d}{dx}f'(x)$.

Using Lagrange notation\index{Lagrange notation}, this is $f''(x)$ or $y''$. You can read this aloud as ``$f$ double prime of $x$'' or ``$y$ double prime.''

Using Leibniz notation\index{Leibniz notation}, the second derivative is written $\dfrac{d^2y}{dx^2}$ or $\dfrac{d^2f}{dx^2}$. This is read aloud as ``the second derivative of $y$ (or $f$).''
\end{definition}
In Section \ref{sec:concavity}, we will discuss geometric interpretations of the second derivative and how it is applied in that context. For now, we will do some examples and highlight a number of important applications related to its conceptual understanding as the rate of change of the derivative of a function.

\begin{example}
Find the second derivative of $f(x)=3x^7$.

\begin{solution} First, we need to find the first derivative:
$$f'(x)=21x^6 \enspace .$$
Then we take the derivative of that function:
$$f''(x)=\frac{d}{dx}f'(x) = \dfrac{d}{dx}21x^6=126x^5 \enspace .$$
\end{solution}\end{example}

\begin{definition}
If $f(x)$ represents the position\index{Position} of a particle at time $x$, then $v(x)=f'(x)$ represents the {\bf velocity}\index{Velocity} (rate of change of the position) of the particle and $a(x)=v'(x)=f''(x)$ represents the {\bf acceleration}\index{Acceleration} (the rate of change of the velocity) of the particle.
\end{definition}

You are probably familiar with acceleration from driving or riding in a car. The speedometer tells you your velocity (speed). When you leave from a stop and press down on the accelerator, you are accelerating -- increasing your speed.

\begin{example}
The height (feet) of a particle after $t$ seconds is $f(t)=t^3-4t^2+8t$ feet. Find the height, velocity and acceleration of the particle when $t= 0, 1,$ and $2$ seconds.

\begin{solution} $f(t)=t^3-4t^2+8t$ so $f(0)=0$ feet, $f(1)=5$ feet, and $f(2)=8$ feet.

The velocity after $t$ seconds is $v(t)=f'(t)=3t^2-8t+8$ feet per second so $v(0)=8$ ft/s, $v(1)=3$ ft/s, and $v(2)=4$ ft/s. At each of these times the velocity is positive and the particle is moving upward, increasing in height.

The acceleration is $a(t)=f''(t)=6t-8$ feet per second per second, or $6t-8$ feet per second squared so $a(0)=-8$ ft/s$^2$, $a(1)=-2$ ft/s$^2$ and $a(2)=4$ ft/s$^2$.

At time 0 and 1, the velocity is positive but the acceleration is negative, so the particle is going up, but its velocity is decreasing at those points; the particle was slowing down. At time 2, the velocity and acceleration are positive, so the particle is speeding up.
\end{solution}\end{example}

% Example: Sales of an iGadget is a logistic curve. Show and interpret second derivative of sales over time, before and after the inflection point.

\subsection{Third and Higher Derivatives}

\subsection{Exercises}
