\protect\hyperlink{main-nav}{≡} \protect\hyperlink{close-nav}{×}

\hypertarget{section-1.4-exponents}{%
\section{Section 1.4: Exponents}\label{section-1.4-exponents}}

The Laws of Exponents let you rewrite algebraic expressions that involve
exponents. The last three listed here are really definitions rather than
rules.

\hypertarget{laws-of-exponents}{%
\paragraph{Laws of Exponents}\label{laws-of-exponents}}

All variables here represent real numbers and all variables in
denominators are nonzero.

\begin{enumerate}
\tightlist
\item
  \textbackslash{}(x\^{}a\textbackslash{}cdot
  x\^{}b=x\^{}\{a+b\}\textbackslash{})
\item
  \textbackslash{}(\textbackslash{}dfrac\{x\^{}a\}\{x\^{}b\}=x\^{}\{a-b\}\textbackslash{})
\item
  \textbackslash{}(\textbackslash{}left(x\^{}a\textbackslash{}right)\^{}b=x\^{}\{ab\}\textbackslash{})
\item
  \textbackslash{}((xy)\^{}a=x\^{}a y\^{}a\textbackslash{})
\item
  \textbackslash{}(\textbackslash{}left(\textbackslash{}dfrac\{x\}\{y\}\textbackslash{}right)\^{}b=\textbackslash{}dfrac\{x\^{}b\}\{y\^{}b\}\textbackslash{})
\item
  \textbackslash{}(x\^{}0=1\textbackslash{}), provided
  \textbackslash{}(x\textbackslash{}neq 0\textbackslash{}). {[}Although
  in some contexts \textbackslash{}(0\^{}0\textbackslash{}) is still
  defined to be 1.{]}
\item
  \textbackslash{}(x\^{}\{-n\}=\textbackslash{}dfrac\{1\}\{x\^{}n\}\textbackslash{}),
  provided \textbackslash{}(x\textbackslash{}neq 0\textbackslash{}).
\item
  \textbackslash{}(x\^{}\{1/n\}=\textbackslash{}sqrt{[}n{]}\{x\}\textbackslash{}),
  provided \textbackslash{}(x\textbackslash{}neq 0\textbackslash{}).
\end{enumerate}

To view this video please enable JavaScript, and consider upgrading to a
web browser that \href{http://videojs.com/html5-video-support/}{supports
HTML5 video}

\hypertarget{example-1}{%
\paragraph{Example 1}\label{example-1}}

Simplify
\textbackslash{}(\textbackslash{}left(2x\^{}2\textbackslash{}right)\^{}3(4x)\textbackslash{}).

We'll begin by simplifying the
\textbackslash{}(\textbackslash{}left(2x\^{}2\textbackslash{}right)\^{}3\textbackslash{})
portion. Using Property 4, we can write

\begin{longtable}[]{@{}ll@{}}
\toprule
\endhead
\textbackslash{}(2\^{}3\textbackslash{}left(x\^{}2\textbackslash{}right)\^{}3(4x)\textbackslash{})
&\tabularnewline
\textbackslash{}(8x\^{}6(4x)\textbackslash{}) & Evaluate
\textbackslash{}(2\^{}3\textbackslash{}), and use Property
3.\tabularnewline
\textbackslash{}(32x\^{}7\textbackslash{}) & Multiply the constants, and
use Property 1, recalling \textbackslash{}(x =
x\^{}1\textbackslash{}).\tabularnewline
\bottomrule
\end{longtable}

Being able to work with negative and fractional exponents will be very
important later in this course.

\hypertarget{example-2}{%
\paragraph{Example 2}\label{example-2}}

Rewrite
\textbackslash{}(\textbackslash{}dfrac\{5\}\{x\^{}3\}\textbackslash{})
using negative exponents.

Since
\textbackslash{}(x\^{}\{-n\}=\textbackslash{}dfrac\{1\}\{x\^{}n\}\textbackslash{}),
then
\textbackslash{}(x\^{}\{-3\}=\textbackslash{}dfrac\{1\}\{x\^{}3\}\textbackslash{})
and thus
\textbackslash{}{[}\textbackslash{}dfrac\{5\}\{x\^{}3\}=5x\^{}\{-3\}.\textbackslash{}{]}

\hypertarget{example-3}{%
\paragraph{Example 3}\label{example-3}}

Simplify
\textbackslash{}(\textbackslash{}left(\textbackslash{}dfrac\{x\^{}\{-2\}\}\{y\^{}\{-3\}\}\textbackslash{}right)\^{}2\textbackslash{})
as much as possible and write your answer using only positive exponents.
\textbackslash{}begin\{align*\}
\textbackslash{}left(\textbackslash{}dfrac\{x\^{}\{-2\}\}\{y\^{}\{-3\}\}\textbackslash{}right)\^{}2=\&
\textbackslash{}dfrac\{\textbackslash{}left(x\^{}\{-2\}\textbackslash{}right)\^{}2\}\{\textbackslash{}left(y\^{}\{-3\}\textbackslash{}right)\^{}2\}\textbackslash{}\textbackslash{}
=\&
\textbackslash{}dfrac\{x\^{}\{-4\}\}\{y\^{}\{-6\}\}\textbackslash{}\textbackslash{}
=\& \textbackslash{}dfrac\{y\^{}6\}\{x\^{}4\}
\textbackslash{}end\{align*\}

\hypertarget{example-4}{%
\paragraph{Example 4}\label{example-4}}

Rewrite
\textbackslash{}(4\textbackslash{}sqrt\{x\}-\textbackslash{}dfrac\{3\}\{\textbackslash{}sqrt\{x\}\}\textbackslash{})
using exponents.

A square root is a radical with index of two. In other words,
\textbackslash{}(\textbackslash{}sqrt\{x\}=\textbackslash{}sqrt{[}2{]}\{x\}\textbackslash{}).
Using the exponent rule above,
\textbackslash{}(\textbackslash{}sqrt\{x\}=\textbackslash{}sqrt{[}2{]}\{x\}=x\^{}\{1/2\}\textbackslash{}).
Rewriting the square roots using the fractional exponent,
\textbackslash{}{[}4\textbackslash{}sqrt\{x\}-\textbackslash{}dfrac\{3\}\{\textbackslash{}sqrt\{x\}\}=4x\^{}\{1/2\}-\textbackslash{}dfrac\{3\}\{x\^{}\{1/2\}\}.\textbackslash{}{]}

Now we can use the negative exponent rule to rewrite the second term in
the expression:
\textbackslash{}{[}4x\^{}\{1/2\}-\textbackslash{}dfrac\{3\}\{x\^{}\{1/2\}\}=4x\^{}\{1/2\}-3x\^{}\{-1/2\}.\textbackslash{}{]}

\hypertarget{example-5}{%
\paragraph{Example 5}\label{example-5}}

Rewrite \textbackslash{}(
\textbackslash{}left(\textbackslash{}sqrt\{p\^{}5\}\textbackslash{}right)\^{}\{-1/3\}
\textbackslash{}) using only positive exponents.

\textbackslash{}begin\{align*\}
\textbackslash{}left(\textbackslash{}sqrt\{p\^{}5\}\textbackslash{}right)\^{}\{-1/3\}=\&
\textbackslash{}left(\textbackslash{}left(p\^{}5\textbackslash{}right)\^{}\{1/2\}\textbackslash{}right)\^{}\{-1/3\}\textbackslash{}\textbackslash{}
=\& p\^{}\{-5/6\}\textbackslash{}\textbackslash{} =\&
\textbackslash{}frac\{1\}\{ p\^{}\{5/6\}\} \textbackslash{}end\{align*\}

\hypertarget{example-6}{%
\paragraph{Example 6}\label{example-6}}

Rewrite \textbackslash{}( x\^{}\{-4/3\} \textbackslash{})as a radical.

\textbackslash{}begin\{align*\} x\^{}\{-4/3\}=\&
\textbackslash{}frac\{1\}\{x\^{}\{4/3\}\}
\textbackslash{}\textbackslash{} =\&
\textbackslash{}frac\{1\}\{\textbackslash{}left(x\^{}\{1/3\}\textbackslash{}right)\^{}4\}
\textbackslash{}quad \textbackslash{}text\{(since
\textbackslash{}(\textbackslash{}frac\{4\}\{3\}=4\textbackslash{}cdot\textbackslash{}frac\{1\}\{3\}\textbackslash{}))\}\textbackslash{}\textbackslash{}
=\&
\textbackslash{}frac\{1\}\{\textbackslash{}left(\textbackslash{}sqrt{[}3{]}\{x\}\textbackslash{}right)\^{}4\}
\textbackslash{}quad \textbackslash{}text\{(using the radical
equivalence)\} \textbackslash{}end\{align*\}

\begin{longtable}[]{@{}ll@{}}
\toprule
\endhead
\href{section1-3.php}{← Previous Section} & \href{section1-5.php}{Next
Section →}\tabularnewline
\bottomrule
\end{longtable}
