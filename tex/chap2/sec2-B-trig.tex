\section{Trigonometric Functions}
\label{sec:trigderiv}

\subsection{Derivative Formulas}

\begin{theorem}
$$\dfrac{d}{dx}\sin(x) = \cos(x)$$

$$\dfrac{d}{dx}\cos(x) = -\sin(x)$$
\end{theorem}

From this, we can figure out the derivatives of the four other trigonometric functions.

\begin{theorem}
$$\dfrac{d}{dx}\tan(x) = \sec^2(x)$$
\end{theorem}
\begin{proof}
Here, we'll use the fact that $\tan(x) = \dfrac{\sin(x)}{\cos(x)}$ and use the Quotient Rule, since we are finding the derivative of a quotient (fraction).
\begin{align*}
    \dfrac{d}{dx}\tan(x) &= \dfrac{d}{dx}\frac{\sin(x)}{\cos(x)} \\
    &= \dfrac{\cos(x)\cdot\frac{d}{dx}\sin(x) - \sin(x)\cdot\frac{d}{dx}\cos(x)}{\cos^2(x)} \\
    &= \dfrac{\cos(x)\cdot\cos(x) - \sin(x)\cdot(-\sin(x))}{\cos^2(x)} \\
    &= \dfrac{\cos^2(x) + \sin^2(x)}{\cos^2(x)} \\
    &= \dfrac{1}{\cos^2(x)}\\
    &= \sec^2(x)
\end{align*}
\end{proof}

\subsection{Examples}

\subsection{Exercises}
