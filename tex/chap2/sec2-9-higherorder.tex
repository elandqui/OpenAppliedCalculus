\section{Higher Order Derivatives}
\label{sec:higherorder}

\subsection{Second Derivative}
\label{ssec:second-deriv}

Let y=f(x). The second derivative of f is the derivative of y′=f′(x).

Using prime notation, this is f′′(x) or y′′. You can read this aloud as "f double prime of x" or "y double prime."

Using Leibniz notation, the second derivative is written d2ydx2 or d2fdx2. This is read aloud as "the second derivative of y (or f)."

If f′′(x) is positive on an interval, the graph of y=f(x) is concave up on that interval. We can say that f is increasing (or decreasing) at an increasing rate.

If f′′(x) is negative on an interval, the graph of y=f(x) is concave down on that interval. We can say that f is increasing (or decreasing) at a decreasing rate.

\begin{example}
Find f′′(x) for f(x)=3x7.

\begin{solution} First, we need to find the first derivative:
f′(x)=21x6.
Then we take the derivative of that function:
f′′(x)=ddx(f′(x))=ddx(21x6)=126x5.
\end{solution}\end{example}

If f(x) represents the position of a particle at time x, then v(x)=f′(x) will represent the velocity (rate of change of the position) of the particle and a(x)=v′(x)=f′′(x) will represent the acceleration (the rate of change of the velocity) of the particle.

You are probably familiar with acceleration from driving or riding in a car. The speedometer tells you your velocity (speed). When you leave from a stop and press down on the accelerator, you are accelerating – increasing your speed.

\begin{example}
The height (feet) of a particle at time t seconds is f(t)=t3–4t2+8t. Find the height, velocity and acceleration of the particle when t= 0, 1, and 2 seconds.

\begin{solution} f(t)=t3–4t2+8t so f(0)=0 feet, f(1)=5 feet, and f(2)=8 feet.

The velocity is v(t)=f′(t)=3t2–8t+8 so v(0)=8 ft/s, v(1)=3 ft/s, and v(2)=4 ft/s. At each of these times the velocity is positive and the particle is moving upward, increasing in height.

The acceleration is a(t)=f′′(t)=6t–8 so a(0)=–8 ft/s2, a(1)=–2 ft/s2 and a(2)=4 ft/s2.

At time 0 and 1, the acceleration is negative, so the particle's velocity would be decreasing at those points - the particle was slowing down. At time 2, the velocity is positive, so the particle was increasing in speed.
\end{solution}\end{example}
