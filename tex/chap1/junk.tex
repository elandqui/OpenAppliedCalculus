\item Dave leaves his office in Padelford Hall on his way to teach in Gould
  Hall. Below are several different scenarios. In each case, sketch a
  plausible (reasonable) graph of the function $s = d(t)$ which keeps track of Dave's distance $s$ from
  Padelford Hall at time $t$. Take distance units to be ``feet''
  and time units to be ``minutes.'' Assume Dave's path to Gould Hall is
  long a straight line which is 2400 feet long. %[UW]

  \includegraphics[width=5.12500in,height=0.92708in]{media/image167.png}


\begin{enumerate}
\def\labelenumi{\alph{enumi}.}
\item
  Dave leaves Padelford Hall and walks at a constant spend until he
  reaches Gould Hall 10 minutes later.
\item
  Dave leaves Padelford Hall and walks at a constant speed. It takes him
  6 minutes to reach the half-way point. Then he gets confused and stops
  for 1 minute. He then continues on to Gould Hall at the same constant
  speed he had when he originally left Padelford Hall.
\item
  Dave leaves Padelford Hall and walks at a constant speed. It takes him
  6 minutes to reach the half-way point. Then he gets confused and stops
  for 1 minute to figure out where he is. Dave then continues on to
  Gould Hall at twice the constant speed he had when he originally left
  Padelford Hall.
\item
  Dave leaves Padelford Hall and walks at a constant speed. It takes him
  6 minutes to reach the half-way point. Then he gets confused and stops
  for 1 minute to figure out where he is. Dave is totally lost, so he
  simply heads back to his office, walking the same constant speed he
  had when he originally left Padelford Hall.
\item
  Dave leaves Padelford heading for Gould Hall at the same instant
  Angela leaves Gould Hall heading for Padelford Hall. Both walk at a
  constant speed, but Angela walks twice as fast as Dave. Indicate a
  plot of ``distance from Padelford'' vs. ``time'' for the both Angela
  and Dave.
\item
  Suppose you want to sketch the graph of a new function s = g(t) that
  keeps track of Dave's distance s from Gould Hall at time t. How would
  your graphs change in (a)-(e)?
\end{enumerate}



Try it Now

\begin{enumerate}
\def\labelenumi{\arabic{enumi}.}
\item
  \begin{quote}
  The population of a small town in the year 1960 was 100 people. Since
  then the population has grown to 1400 people reported during the 2010
  census. Choose descriptive variables for your input and output and use
  interval notation to write the domain and range.
  \end{quote}
\end{enumerate}


Example 11

Express the relationship as a function $y = f(x)$ if possible.

If we try to solve for $y$ in this equation:

We end up with two outputs corresponding to the same input, so this
relationship cannot be represented as a single function $y = f(x)$.

As with tables and graphs, it is common to evaluate and solve functions
involving formulas. ~Evaluating will require replacing the input
variable in the formula with the value provided and calculating.
~Solving will require replacing the output variable in the formula with
the value provided, and solving for the input(s) that would produce that
output.

Example 12

Given the function

a) Evaluate $k}(2)

b) Solve $k(t)} = 1

a) To evaluate $k}(2), we plug in the input value 2 into the
formula wherever we see the input variable $t}, then simplify

So $k}(2) = 10

b) To solve $k(t)} = 1, we set the formula for $k(t)} equal to
1, and solve for the input value that will produce that output

$k(t)} = 1 substitute the original formula

subtract 2 from each side

take the cube root of each side

When solving an equation using formulas, you can check your answer by
using your solution in the original equation to see if your calculated
answer is correct.

We want to know if is true when .

=

= 1 which was the desired result.

Example 13

Given the function

a) Evaluate $h$(4)

b) Solve $h(p)} = 3

To evaluate $h$(4) we substitute the value 4 for the input variable
$p} in the given function.

a)

= 16 + 8

= 24

b) $h(p)} = 3 Substitute the original function

This is quadratic, so we can rearrange the equation to get it = 0

subtract 3 from each side

this is factorable, so we factor it

By the zero factor theorem since , either or (or both of them equal 0)
and so we solve both equations for $p}, finding $p} = -3 from
the first equation and $p} = 1 from the second equation.

This gives us the solution: $h(p)} = 3 when $p} = 1 or
$p} = -3~

We found two solutions in this case, which tells us this function is not
one-to-one.

Try it Now

6. Given the function\\
a. Evaluate $g}(5)\\
b. Solve $g(m)} = 2

\textbf{Basic Toolkit Functions}

In this text, we will be exploring functions -- the shapes of their
graphs, their unique features, their equations, and how to solve
problems with them. When learning to read, we start with the alphabet.
When learning to do arithmetic, we start with numbers. When working with
functions, it is similarly helpful to have a base set of elements to
build from. We call these our ``toolkit of functions'' -- a set of basic
named functions for which we know the graph, equation, and special
features.

For these definitions we will use $x} as the input variable and
$f(x)} as the output variable.

Toolkit Functions

Linear

Constant: , where $c} is a constant (number)

Identity:

Absolute Value:

Power

Quadratic:

Cubic:

Reciprocal:

Reciprocal squared:

Square root:

Cube root:

You will see these toolkit functions, combinations of toolkit functions,
their graphs and their transformations frequently throughout this book.
In order to successfully follow along later in the book, it will be very
helpful if you can recognize these toolkit functions and their features
quickly by name, equation, graph and basic table values.

Not every important equation can be written as $y$ = $f(x)}.
An example of this is the equation of a circle. Recall the distance
formula for the distance between two points:

A circle with radius $r} with center at ($h$, $k}) can be
described as all points ($x}, $y$) a distance of $r} from
the center, so using the distance formula, , giving

Equation of a circle

A circle with radius $r} with center ($h$, $k}) has
equation

\textbf{\\
}

% \textbf{Graphs of the Toolkit Functions}
%
% Constant Function: Identity: Absolute Value:
%
% \includegraphics[width=1.77481in,height=1.75000in]{media/image42.png}
% \protect\hypertarget{_Hlk481683299}{}{}\includegraphics[width=1.71429in,height=1.75000in]{media/image43.png}
% \includegraphics[width=1.76512in,height=1.75000in]{media/image44.png}
%
% Quadratic: Cubic: Square root:
%
% \includegraphics[width=1.71682in,height=1.75000in]{media/image46.png}
% \includegraphics[width=1.71701in,height=1.75000in]{media/image47.png}
% \includegraphics[width=1.72879in,height=1.75000in]{media/image48.png}
%
% Cube root: Reciprocal: Reciprocal squared:
%
% \includegraphics[width=1.73773in,height=1.75000in]{media/image49.png}
% \includegraphics[width=1.69659in,height=1.75000in]{media/image50.png}
% \includegraphics[width=1.74596in,height=1.75000in]{media/image51.png}


Try it Now Answers

1. Yes: for each bank account, there would be one balance associated

2. No: there could be several bank accounts with the same balance

3. Yes it's a function; No, it's not one-to-one (several percents give
the same letter grade)

4. When $N$=4, $Q}=$g}(4)=6

5. There are two points where the output is 1: $x} = 0 or $x}
= 2

6. a.\\
b. . Square both sides to get . $m} = 8

textbf{Notation}

In the previous examples, we used inequalities to describe the domain
and range of the functions. This is one way to describe intervals of
input and output values, but is not the only way. Let us take a moment
to discuss notation for domain and range.

Using inequalities, such as , , and imply that we are interested in all
values between the low and high values, including the high values in
these examples.

However, occasionally we are interested in a specific list of numbers
like the range for the price to send letters, $p} = \$0.44, \$0.61,
\$0.78, or \$0.95. These numbers represent a set of specific values:
\{0.44, 0.61, 0.78, 0.95\}

Representing values as a set, or giving instructions on how a set is
built, leads us to another type of notation to describe the domain and
range.

Suppose we want to describe the values for a variable $x} that are
10 or greater, but less than 30. In inequalities, we would write .

When describing domains and ranges, we sometimes extend this into
\textbf{set-builder notation}, which would look like this: . The curly
brackets \{\} are read as ``the set of'', and the vertical bar
\textbar{} is read as ``such that'', so altogether we would read as
``the set of $x}-values such that 10 is less than or equal to
$x} and $x} is less than 30.''

When describing ranges in set-builder notation, we could similarly write
something like , or if the output had its own variable, we could use it.
So for our tree height example above, we could write for the range . In
set-builder notation, if a domain or range is not limited, we could
write , or , read as ``the set of $t}-values such that $t} is
an element of the set of real numbers.





Remember when writing or reading interval notation:

Using a square bracket {[} means the start value is included in the set

Using a parenthesis ( means the start value is not included in the set

Try it Now

2. Given the following interval, write its meaning in words, set builder
notation, and interval notation.

\includegraphics[width=4.16667in,height=0.52083in]{media/image197.png}

\textbf{Domain and Range from Graphs}

We can also talk about domain and range based on graphs. Since domain
refers to the set of possible input values, the domain of a graph
consists of all the input values shown on the graph. Remember that input
values are almost always shown along the horizontal axis of the graph.
Likewise, since range is the set of possible output values, the range of
a graph we can see from the possible values along the vertical axis of
the graph.

Be careful -- if the graph continues beyond the window on which we can
see the graph, the domain and range might be larger than the values we
can see.

Example 4

Determine the domain and range of the graph below.

\includegraphics[width=3.75000in,height=3.48958in]{media/image198.png}

In the graph above\footnote{\url{http://commons.wikimedia.org/wiki/File:Alaska_Crude_Oil_Production.PNG},
  CC-BY-SA, July 19, 2010}, the input quantity along the horizontal axis
appears to be ``year'', which we could notate with the variable
$y$. The output is ``thousands of barrels of oil per day'', which
we might notate with the variable $b}, for barrels. The graph would
likely continue to the left and right beyond what is shown, but based on
the portion of the graph that is shown to us, we can determine the
domain is , and the range is approximately.

In interval notation, the domain would be {[}1975, 2008{]} and the range
would be about {[}180, 2010{]}. For the range, we have to approximate
the smallest and largest outputs since they don't fall exactly on the
grid lines.

Remember that, as in the previous example, $x} and $y$ are not
always the input and output variables. Using descriptive variables is an
important tool to remembering the context of the problem.

Try it Now

3. Given the graph below write the domain and range in interval notation

\includegraphics[width=3.84375in,height=2.62500in]{media/image201.png}

\textbf{Domains and Ranges of the Toolkit functions}

We will now return to our set of toolkit functions to note the domain
and range of each.

$Constant Function}:

The domain here is not restricted; $x} can be anything. When this
is the case we say the domain is all real numbers. The outputs are
limited to the constant value of the function.

Domain:

Range: {[}$c}{]}

$Since there is only one output value, we list it by itself in
square brackets.}

$Identity Function}:

Domain:

Range:

$Quadratic Function}:

Domain:

Range:

$Multiplying a negative or positive number by itself can only yield
a positive output. }

$Cubic Function}:

Domain:

Range:

$Reciprocal}:

Domain:

Range:

$We cannot divide by 0 so we must exclude 0 from the domain.}

$One divide by any value can never be 0, so the range will not
include 0.}

$Reciprocal squared}:

Domain:

Range:

$We cannot divide by 0 so we must exclude 0 from the domain. }

$Cube Root}:

Domain:

Range:

$Square Root}: , commonly just written as,

Domain:

Range:

$When dealing with the set of real numbers we cannot take the
square root of a negative number so the domain is limited to 0 or
greater.}

$Absolute Value Function}:

Domain:

Range:

$Since absolute value is defined as a distance from 0, the output
can only be greater than or equal to 0.}


\textbf{Piecewise Functions}

In the toolkit functions we introduced the absolute value function .

With a domain of all real numbers and a range of values greater than or
equal to 0, the absolute value can be defined as the magnitude or
modulus of a number, a real number value regardless of sign, the size of
the number, or the distance from 0 on the number line. All of these
definitions require the output to be greater than or equal to 0.

If we input 0, or a positive value the output is unchanged

if

If we input a negative value the sign must change from negative to
positive.

if , since multiplying a negative value by -1 makes it positive.

Since this requires two different processes or pieces, the absolute
value function is often called the most basic piecewise defined
function.

Piecewise Function

A \textbf{piecewise function} is a function in which the formula used
depends upon the domain the input lies in. We notate this idea like:

Example 6

A museum charges \$5 per person for a guided tour with a group of 1 to 9
people, or a fixed \$50 fee for 10 or more people in the group. Set up a
function relating the number of people, $N$, to the cost, $C}.

To set up this function, two different formulas would be needed.
$C} = 5$N$ would work for $N$ values under 10, and
$C} = 50 would work for values of $N$ ten or greater. Notating
this:

Example 7

A cell phone company uses the function below to determine the cost,
$C}, in dollars for $g} gigabytes of data transfer.

Find the cost of using 1.5 gigabytes of data, and the cost of using 4
gigabytes of data.

To find the cost of using 1.5 gigabytes of data, $C}(1.5), we first
look to see which piece of domain our input falls in. Since 1.5 is less
than 2, we use the first formula, giving $C}(1.5) = \$25.

The find the cost of using 4 gigabytes of data, $C}(4), we see that
our input of 4 is greater than 2, so we'll use the second formula.
$C}(4) = 25 + 10(4-2) = \$45.

Example 8

Sketch a graph of the function

The first two component functions are from our library of Toolkit
functions, so we know their shapes. We can imagine graphing each
function, then limiting the graph to the indicated domain. At the
endpoints of the domain, we put open circles to indicate where the
endpoint is not included, due to a strictly-less-than inequality, and a
closed circle where the endpoint is included, due to a
less-than-or-equal-to inequality.

\includegraphics[width=1.79063in,height=1.80000in]{media/image228.png}
.\includegraphics[width=1.80628in,height=1.80000in]{media/image229.png}

\includegraphics[width=1.81582in,height=1.80000in]{media/image230.png}

For the third function, you should recognize this as a linear equation
from your previous coursework. If you remember how to graph a line using
slope and intercept, you can do that. Otherwise, we could calculate a
couple values, plot points, and connect them with a line.

At $x} = 2, $f$(2) = 6 -- 2 = 4. We place an open circle
here.\\
At $x} = 3, $f$(3) = 6 -- 3 = 3. Connect these points with a
line.

Now that we have each piece individually, we combine them onto the same
graph:

\includegraphics[width=2.20385in,height=2.20000in]{media/image231.png}

Try it Now

4. At Pierce College during the 2009-2010 school year tuition rates for
in-state residents were \$89.50 per credit for the first 10 credits,
\$33 per credit for credits 11-18, and for over 18 credits the rate is
\$73 per credit\footnote{\url{https://www.pierce.ctc.edu/dist/tuition/ref/files/0910_tuition_rate.pdf},
  retrieved August 6, 2010}. Write a piecewise defined function for the
total tuition, $T}, at Pierce College during 2009-2010 as a
function of the number of credits taken, $c}. Be sure to consider a
reasonable domain and range.

Important Topics of this Section

Definition of domain

Definition of range

Inequalities

Interval notation

Set builder notation

Domain and Range from graphs

Domain and Range of toolkit functions

Piecewise defined functions

Try it Now Answers

1. Domain; $y$ = years {[}1960,2010{]} ; Range, $p} =
population, {[}100,1400{]}

2. a. Values that are less than or equal to -2, or values that are
greater than or equal to -1 and less than 3\\
b.\\
c.

3. Domain; $y$=years, {[}1952,2002{]} ; Range, $p}=population
in millions, {[}40,88{]}

4. Tuition, $T}, as a function of credits, $c}.

Reasonable domain should be whole numbers 0 to (answers may vary), e.g.

Reasonable range should be \$0 -- (answers may vary), e.g.

\subsection{Section 1.2 Exercises}\label{section-1.2-exercises}

Write the domain and range of the function using interval notation.

1.
\includegraphics[width=2.34830in,height=1.60000in]{media/image237.png}
2.
\includegraphics[width=1.67032in,height=1.80000in]{media/image238.png}

Write the domain and range of each graph as an inequality.\\
3.
\includegraphics[width=2.02817in,height=2.00000in]{media/image239.png}
4.
\includegraphics[width=2.04019in,height=2.00000in]{media/image240.png}

Suppose that you are holding your toy submarine under the water. You
release it and it begins to ascend. The graph models the depth of the
submarine as a function of time, stopping once the sub surfaces. What is
the domain and range of the function in the graph?

5.
\includegraphics[width=1.93814in,height=1.90000in]{media/image241.png}
6.
\includegraphics[width=1.88690in,height=1.90000in]{media/image242.png}

Find the domain of each function

7. 8.

9. 10.

11. 12.

13. 14.

15. 16.

17. 18.

Given each function, evaluate: , , ,

19. 20.

21. 22.

23. 24.

Write a formula for the piecewise function graphed below.\\
25.\includegraphics[width=2.32252in,height=1.90000in]{media/image265.png}
26.\includegraphics[width=2.27420in,height=1.90000in]{media/image266.png}

27.
\includegraphics[width=2.25854in,height=1.90000in]{media/image267.png}
28.\includegraphics[width=2.28954in,height=1.90000in]{media/image268.png}

29.
\includegraphics[width=2.37431in,height=1.90000in]{media/image269.png}
30.\includegraphics[width=2.24324in,height=1.90000in]{media/image270.png}

Sketch a graph of each piecewise function

31. 32.

33. 34.

35. 36.

\hypertarget{section-1.3-rates-of-change-and-behavior-of-graphs}{\subsection{Section
1.3 Rates of Change and Behavior of
Graphs}\label{section-1.3-rates-of-change-and-behavior-of-graphs}}

Since functions represent how an output quantity varies with an input
quantity, it is natural to ask about the rate at which the values of the
function are changing.

For example, the function $C(t)} below gives the average cost, in
dollars, of a gallon of gasoline $t} years after 2000.

\begin{longtable}[]{@{}lllllllll@{}}
\toprule
$t} & 2 & 3 & 4 & 5 & 6 & 7 & 8 & 9\tabularnewline
\midrule
\endhead
$C(t)} & 1.47 & 1.69 & 1.94 & 2.30 & 2.51 & 2.64 & 3.01 &
2.14\tabularnewline
\bottomrule
\end{longtable}

If we were interested in how the gas prices had changed between 2002 and
2009, we could compute that the cost per gallon had increased from
\$1.47 to \$2.14, an increase of \$0.67. While this is interesting, it
might be more useful to look at how much the price changed $per
year}. You are probably noticing that the price didn't change the same
amount each year, so we would be finding the \textbf{average rate of
change} over a specified amount of time.

The gas price increased by \$0.67 from 2002 to 2009, over 7 years, for
an average of dollars per year. On average, the price of gas increased
by about 9.6 cents each year.

Rate of Change

A \textbf{rate of change} describes how the output quantity changes in
relation to the input quantity. The units on a rate of change are
``$output units} per $input units}''

Some other examples of rates of change would be quantities like:

\begin{itemize}
\item
  A population of rats increases by 40 rats per week
\item
  A barista earns \$9 per hour (dollars per hour)
\item
  A farmer plants 60,000 onions per acre
\item
  A car can drive 27 miles per gallon
\item
  A population of grey whales decreases by 8 whales per year
\item
  The amount of money in your college account decreases by \$4,000 per
  quarter
\end{itemize}

Average Rate of Change

The \textbf{average rate of change} between two input values is the
total change of the function values (output values) divided by the
change in the input values.

Average rate of change = =

Example 1

Using the cost-of-gas function from earlier, find the average rate of
change between 2007 and 2009

From the table, in 2007 the cost of gas was \$2.64. In 2009 the cost was
\$2.14.

The input (years) has changed by 2. The output has changed by \$2.14 -
\$2.64 = -0.50. The average rate of change is then = -0.25 dollars per
year

Try it Now

1. Using the same cost-of-gas function, find the average rate of change
between 2003 and 2008

Notice that in the last example the change of output was $negative}
since the output value of the function had decreased. Correspondingly,
the average rate of change is negative.

Example 2

\includegraphics[width=1.84028in,height=1.82778in]{media/image281.png}Given
the function $g(t)} shown here, find the average rate of change on
the interval\\
{[}0, 3{]}.

At $t} = 0, the graph shows

At $t} = 3, the graph shows

The output has changed by 3 while the input has changed by 3, giving an
average rate of change of:

Example 3

On a road trip, after picking up your friend who lives 10 miles away,
you decide to record your distance from home over time. Find your
average speed over the first 6 hours.

Here, your average speed is the average rate of change.

You traveled 282 miles in 6 hours, for an average speed of

= 47 miles per hour

We can more formally state the average rate of change calculation using
function notation.

Average Rate of Change using Function Notation

Given a function $f(x)}, the average rate of change on the interval
{[}$a$, $b}{]} is

Average rate of change =

Example 4

Compute the average rate of change of on the interval {[}2, 4{]}

We can start by computing the function values at each endpoint of the
interval

Now computing the average rate of change

Average rate of change =

Try it Now

2. Find the average rate of change of on the interval {[}1, 9{]}

Example 5

The magnetic force $f$, measured in Newtons, between two magnets is
related to the distance between the magnets $d$, in centimeters, by
the formula . Find the average rate of change of force if the distance
between the magnets is increased from 2 cm to 6 cm.

We are computing the average rate of change of on the interval {[}2,
6{]}.

Average rate of change = Evaluating the function

=

Simplifying

Combining the numerator terms

Simplifying further

Newtons per centimeter

This tells us the magnetic force decreases, on average, by 1/9 Newtons
per centimeter over this interval.

Example 6

Find the average rate of change of on the interval . Your answer will be
an expression involving $a$.

Using the average rate of change formula

Evaluating the function

Simplifying

Simplifying further, and factoring

Cancelling the common factor $a$

This result tells us the average rate of change between $t} = 0 and
any other point $t} = $a$. For example, on the interval {[}0,
5{]}, the average rate of change would be 5+3 = 8.

Try it Now

3. Find the average rate of change of on the interval .

\textbf{Graphical Behavior of Functions}

As part of exploring how functions change, it is interesting to explore
the graphical behavior of functions.

Increasing/Decreasing

A function is \textbf{increasing} on an interval if the function values
increase as the inputs increase. More formally, a function is increasing
if $f(b)} \textgreater{} $f(a)} for any two input values
$a$ and $b} in the interval with $b\textgreater{}a}. The
average rate of change of an increasing function is \textbf{positive.}

A function is \textbf{decreasing} on an interval if the function values
decrease as the inputs increase. More formally, a function is decreasing
if $f(b)} \textless{} $f(a)} for any two input values $a$
and $b} in the interval with $b\textgreater{}a}. The average
rate of change of a decreasing function is \textbf{negative.}

Example 7

\includegraphics[width=1.84167in,height=1.84722in]{media/image308.png}Given
the function $p(t)} graphed here, on what intervals does the
function appear to be increasing?

The function appears to be increasing from $t} = 1 to $t} = 3,
and from $t} = 4 on.

In interval notation, we would say the function appears to be increasing
on the interval and the interval .

Notice in the last example that we used open intervals (intervals that
don't include the endpoints) since the function is neither increasing
nor decreasing at $t} = 1, 3, or 4.

Local Extrema

A point where a function changes from increasing to decreasing is called
a \textbf{local maximum}.

A point where a function changes from decreasing to increasing is called
a \textbf{local minimum}.

Together, local maxima and minima are called the \textbf{local extrema},
or local extreme values, of the function.

Example 8

Using the cost of gasoline function from the beginning of the section,
find an interval on which the function appears to be decreasing.
Estimate any local extrema using the table.

It appears that the cost of gas increased from $t} = 2 to $t}
= 8. It appears the cost of gas decreased from $t} = 8 to $t}
= 9, so the function appears to be decreasing on the interval (8, 9).

Since the function appears to change from increasing to decreasing at
$t} = 8, there is local maximum at $t} = 8.

Example 9

Use a graph to estimate the local extrema of the function . Use these to
determine the intervals on which the function is increasing.

Using technology to graph the function, it appears there is a local
minimum somewhere between $x} = 2 and $x} =3, and a symmetric
local maximum somewhere between $x} = -3 and $x} = -2.

Most graphing calculators and graphing utilities can estimate the
location of maxima and minima. Below are screen images from two
different technologies, showing the estimate for the local maximum and
minimum.

\includegraphics[width=2.11458in,height=1.55208in]{media/image312.png}
\includegraphics[width=2.26528in,height=1.95852in]{media/image313.png}

Based on these estimates, the function is increasing on the intervals
and . Notice that while we expect the extrema to be symmetric, the two
different technologies agree only up to 4 decimals due to the differing
approximation algorithms used by each.

Try it Now

4. Use a graph of the function to estimate the local extrema of the
function. Use these to determine the intervals on which the function is
increasing and decreasing.

\textbf{Concavity}

The total sales, in thousands of dollars, for two companies over 4 weeks
are shown.

\includegraphics[width=2.21495in,height=2.20000in]{media/image317.png}
\includegraphics[width=2.16615in,height=2.20000in]{media/image318.png}

Company A Company B

As you can see, the sales for each company are increasing, but they are
increasing in very different ways. To describe the difference in
behavior, we can investigate how the average rate of change varies over
different intervals. Using tables of values,

From the tables, we can see that the rate of change for company A is
$decreasing}, while the rate of change for company B is
$increasing}.

\includegraphics[width=2.19525in,height=2.20000in]{media/image319.png}
\includegraphics[width=2.19566in,height=2.20000in]{media/image320.png}

When the rate of change is getting smaller, as with Company A, we say
the function is \textbf{concave down}. When the rate of change is
getting larger, as with Company B, we say the function is
\textbf{concave up}.

Concavity

A function is \textbf{concave up} if the rate of change is increasing.

A function is \textbf{concave down} if the rate of change is decreasing.

A point where a function changes from concave up to concave down or vice
versa is called an \textbf{inflection point}.

Example 10

\includegraphics[width=1.60714in,height=2.49806in]{media/image321.png}An
object is thrown from the top of a building. The object's height in feet
above ground after $t} seconds is given by the function for .
Describe the concavity of the graph.

Sketching a graph of the function, we can see that the function is
decreasing. We can calculate some rates of change to explore the
behavior.

Notice that the rates of change are becoming more negative, so the rates
of change are $decreasing}. This means the function is concave
down.

Example 11

The value, $V}, of a car after $t} years is given in the table
below. Is the value increasing or decreasing? Is the function concave up
or concave down?

Since the values are getting smaller, we can determine that the value is
decreasing. We can compute rates of change to determine concavity.

Since these values are becoming less negative, the rates of change are
$increasing}, so this function is concave up.

Try it Now

5. Is the function described in the table below concave up or concave
down?

Graphically, concave down functions bend downwards like a frown, and
concave up function bend upwards like a smile.

Example 12

\includegraphics[width=2.71597in,height=2.35069in]{media/image324.png}Estimate
from the graph shown the intervals on which the function is concave down
and concave up.

On the far left, the graph is decreasing but concave up, since it is
bending upwards. It begins increasing at $x} = -2, but it continues
to bend upwards until about $x} = -1.

From $x} = -1 the graph starts to bend downward, and continues to
do so until about $x} = 2. The graph then begins curving upwards
for the remainder of the graph shown.

From this, we can estimate that the graph is concave up on the intervals
and , and is concave down on the interval . The graph has inflection
points at $x} = -1 and $x} = 2.

Try it Now

6. Using the graph from Try it Now 4, , estimate the intervals on which
the function is concave up and concave down.

\textbf{Behaviors of the Toolkit Functions}

We will now return to our toolkit functions and discuss their graphical
behavior.

\begin{longtable}[]{@{}lll@{}}
\toprule
Function & Increasing/Decreasing & Concavity\tabularnewline
\midrule
\endhead
$Constant Function} & Neither increasing nor decreasing & Neither
concave up nor down\tabularnewline
$Identity Function} & Increasing & Neither concave up nor
down\tabularnewline
\begin{minipage}[t]{0.32\columnwidth}\raggedright\strut
$Quadratic Function}\strut
\end{minipage} & \begin{minipage}[t]{0.32\columnwidth}\raggedright\strut
Increasing on

Decreasing on

Minimum at $x} = 0\strut
\end{minipage} & \begin{minipage}[t]{0.32\columnwidth}\raggedright\strut
Concave up\strut
\end{minipage}\tabularnewline
\begin{minipage}[t]{0.32\columnwidth}\raggedright\strut
$Cubic Function }\strut
\end{minipage} & \begin{minipage}[t]{0.32\columnwidth}\raggedright\strut
Increasing\strut
\end{minipage} & \begin{minipage}[t]{0.32\columnwidth}\raggedright\strut
Concave down on

Concave up on

Inflection point at (0,0)\strut
\end{minipage}\tabularnewline
\begin{minipage}[t]{0.32\columnwidth}\raggedright\strut
$Reciprocal}\strut
\end{minipage} & \begin{minipage}[t]{0.32\columnwidth}\raggedright\strut
Decreasing\strut
\end{minipage} & \begin{minipage}[t]{0.32\columnwidth}\raggedright\strut
Concave down on

Concave up on\strut
\end{minipage}\tabularnewline
Function & Increasing/Decreasing & Concavity\tabularnewline
\begin{minipage}[t]{0.32\columnwidth}\raggedright\strut
$Reciprocal squared}\strut
\end{minipage} & \begin{minipage}[t]{0.32\columnwidth}\raggedright\strut
Increasing on

Decreasing on\strut
\end{minipage} & \begin{minipage}[t]{0.32\columnwidth}\raggedright\strut
Concave up on\strut
\end{minipage}\tabularnewline
\begin{minipage}[t]{0.32\columnwidth}\raggedright\strut
$Cube Root}\strut
\end{minipage} & \begin{minipage}[t]{0.32\columnwidth}\raggedright\strut
Increasing\strut
\end{minipage} & \begin{minipage}[t]{0.32\columnwidth}\raggedright\strut
Concave down on

Concave up on

Inflection point at (0,0)\strut
\end{minipage}\tabularnewline
$Square Root} & Increasing on & Concave down on\tabularnewline
\begin{minipage}[t]{0.32\columnwidth}\raggedright\strut
$Absolute Value}\strut
\end{minipage} & \begin{minipage}[t]{0.32\columnwidth}\raggedright\strut
Increasing on

Decreasing on\strut
\end{minipage} & \begin{minipage}[t]{0.32\columnwidth}\raggedright\strut
Neither concave up or down\strut
\end{minipage}\tabularnewline
\bottomrule
\end{longtable}

Important Topics of This Section

Rate of Change

Average Rate of Change

Calculating Average Rate of Change using Function Notation

Increasing/Decreasing

Local Maxima and Minima (Extrema)

Inflection points

Concavity

Try it Now Answers

1. = 0.264 dollars per year.

2. Average rate of change =

3.

\includegraphics[width=1.88264in,height=1.42639in]{media/image337.png}

4. Based on the graph, the local maximum appears to occur at (-1, 28),
and the local minimum occurs at (5,-80). The function is increasing on
and decreasing on .

5. Calculating the rates of change, we see the rates of change become
$more} negative, so the rates of change are $decreasing}. This
function is concave down.

6. Looking at the graph, it appears the function is concave down on and
concave up on .

\subsection{Section 1.3 Exercises}\label{section-1.3-exercises}

1. The table below gives the annual sales (in millions of dollars) of a
product. What was the average rate of change of annual sales\ldots{}\\
a) Between 2001 and 2002? b) Between 2001 and 2004?

\begin{longtable}[]{@{}llllllllll@{}}
\toprule
\textbf{year} & 1998 & 1999 & 2000 & 2001 & 2002 & 2003 & 2004 & 2005 &
2006\tabularnewline
\midrule
\endhead
\textbf{sales} & 201 & 219 & 233 & 243 & 249 & 251 & 249 & 243 &
233\tabularnewline
\bottomrule
\end{longtable}

2. The table below gives the population of a town, in thousands. What
was the average rate of change of population\ldots{}\\
a) Between 2002 and 2004? b) Between 2002 and 2006?

\begin{longtable}[]{@{}llllllllll@{}}
\toprule
\textbf{year} & 2000 & 2001 & 2002 & 2003 & 2004 & 2005 & 2006 & 2007 &
2008\tabularnewline
\midrule
\endhead
\textbf{population} & 87 & 84 & 83 & 80 & 77 & 76 & 75 & 78 &
81\tabularnewline
\bottomrule
\end{longtable}

\includegraphics[width=1.74306in,height=1.70972in]{media/image342.png}

3. Based on the graph shown, estimate the average rate of change from
$x} = 1 to $x} = 4.

4. Based on the graph shown, estimate the average rate of change from
$x} = 2 to $x} = 5.

Find the average rate of change of each function on the interval
specified.

5. on {[}1, 5{]} 6. on {[}-4, 2{]}

7. on {[}-3, 3{]} 8. on {[}-2, 4{]}

9. on {[}-1, 3{]} 10. on {[}-3, 1{]}

Find the average rate of change of each function on the interval
specified. Your answers will be expressions involving a parameter
($b} or $h$).

11. on {[}1, $b}{]} 12. on {[}4, $b}{]}

13. on {[}2, 2+$h${]} 14. on {[}3, 3+$h${]}

15. on {[}9, 9+$h${]} 16. on {[}1, 1+$h${]}

17. on {[}1, 1+$h${]} 18. on {[}2, 2+$h${]}

19. on {[}$x}, $x}+$h${]} 20. on {[}$x},
$x}+$h${]}

For each function graphed, estimate the intervals on which the function
is increasing and decreasing.

\includegraphics[width=2.28681in,height=2.11736in]{media/image359.png}21.
\includegraphics[width=2.34722in,height=2.12917in]{media/image360.png}
22.

\includegraphics[width=2.61111in,height=1.99653in]{media/image361.png}\includegraphics[width=1.93681in,height=1.93681in]{media/image362.png}23.
24.

For each table below, select whether the table represents a function
that is increasing or decreasing, and whether the function is concave up
or concave down.

\begin{longtable}[]{@{}llllllll@{}}
\toprule
\begin{minipage}[b]{0.12\columnwidth}\raggedright\strut
25.\strut
\end{minipage} & \begin{minipage}[b]{0.12\columnwidth}\raggedright\strut
\begin{longtable}[]{@{}ll@{}}
\toprule
$\textbf{x}} & $\textbf{f(x)}}\tabularnewline
\midrule
\endhead
1 & 2\tabularnewline
2 & 4\tabularnewline
3 & 8\tabularnewline
4 & 16\tabularnewline
5 & 32\tabularnewline
\bottomrule
\end{longtable}\strut
\end{minipage} & \begin{minipage}[b]{0.12\columnwidth}\raggedright\strut
26.\strut
\end{minipage} & \begin{minipage}[b]{0.12\columnwidth}\raggedright\strut
\begin{longtable}[]{@{}ll@{}}
\toprule
$\textbf{x}} & $\textbf{g(x)}}\tabularnewline
\midrule
\endhead
1 & 90\tabularnewline
2 & 80\tabularnewline
3 & 75\tabularnewline
4 & 72\tabularnewline
5 & 70\tabularnewline
\bottomrule
\end{longtable}\strut
\end{minipage} & \begin{minipage}[b]{0.12\columnwidth}\raggedright\strut
27.\strut
\end{minipage} & \begin{minipage}[b]{0.12\columnwidth}\raggedright\strut
\begin{longtable}[]{@{}ll@{}}
\toprule
$\textbf{x}} & $\textbf{h(x)}}\tabularnewline
\midrule
\endhead
1 & 300\tabularnewline
2 & 290\tabularnewline
3 & 270\tabularnewline
4 & 240\tabularnewline
5 & 200\tabularnewline
\bottomrule
\end{longtable}\strut
\end{minipage} & \begin{minipage}[b]{0.12\columnwidth}\raggedright\strut
28.\strut
\end{minipage} & \begin{minipage}[b]{0.12\columnwidth}\raggedright\strut
\begin{longtable}[]{@{}ll@{}}
\toprule
$\textbf{x}} & $\textbf{k(x)}}\tabularnewline
\midrule
\endhead
1 & 0\tabularnewline
2 & 15\tabularnewline
3 & 25\tabularnewline
4 & 32\tabularnewline
5 & 35\tabularnewline
\bottomrule
\end{longtable}\strut
\end{minipage}\tabularnewline
\midrule
\endhead
& & & & & & &\tabularnewline
\begin{minipage}[t]{0.12\columnwidth}\raggedright\strut
29.\strut
\end{minipage} & \begin{minipage}[t]{0.12\columnwidth}\raggedright\strut
\begin{longtable}[]{@{}ll@{}}
\toprule
$\textbf{x}} & $\textbf{f(x)}}\tabularnewline
\midrule
\endhead
1 & -10\tabularnewline
2 & -25\tabularnewline
3 & -37\tabularnewline
4 & -47\tabularnewline
5 & -54\tabularnewline
\bottomrule
\end{longtable}\strut
\end{minipage} & \begin{minipage}[t]{0.12\columnwidth}\raggedright\strut
30.\strut
\end{minipage} & \begin{minipage}[t]{0.12\columnwidth}\raggedright\strut
\begin{longtable}[]{@{}ll@{}}
\toprule
$\textbf{x}} & $\textbf{g(x)}}\tabularnewline
\midrule
\endhead
1 & -200\tabularnewline
2 & -190\tabularnewline
3 & -160\tabularnewline
4 & -100\tabularnewline
5 & 0\tabularnewline
\bottomrule
\end{longtable}\strut
\end{minipage} & \begin{minipage}[t]{0.12\columnwidth}\raggedright\strut
31.\strut
\end{minipage} & \begin{minipage}[t]{0.12\columnwidth}\raggedright\strut
\begin{longtable}[]{@{}ll@{}}
\toprule
$\textbf{x}} & $\textbf{h(x)}}\tabularnewline
\midrule
\endhead
1 & -100\tabularnewline
2 & -50\tabularnewline
3 & -25\tabularnewline
4 & -10\tabularnewline
5 & 0\tabularnewline
\bottomrule
\end{longtable}\strut
\end{minipage} & \begin{minipage}[t]{0.12\columnwidth}\raggedright\strut
32.\strut
\end{minipage} & \begin{minipage}[t]{0.12\columnwidth}\raggedright\strut
\begin{longtable}[]{@{}ll@{}}
\toprule
$\textbf{x}} & $\textbf{k(x)}}\tabularnewline
\midrule
\endhead
1 & -50\tabularnewline
2 & -100\tabularnewline
3 & -200\tabularnewline
4 & -400\tabularnewline
5 & -900\tabularnewline
\bottomrule
\end{longtable}\strut
\end{minipage}\tabularnewline
& & & & & & &\tabularnewline
\bottomrule
\end{longtable}

\includegraphics[width=2.45278in,height=2.29931in]{media/image363.png}\includegraphics[width=2.39514in,height=2.22500in]{media/image364.png}For
each function graphed, estimate the intervals on which the function is
concave up and concave down, and the location of any inflection points.

\includegraphics[width=2.41736in,height=2.19931in]{media/image365.png}33.
34.

35.
\includegraphics[width=2.60244in,height=1.99028in]{media/image361.png}36.

Use a graph to estimate the local extrema and inflection points of each
function, and to estimate the intervals on which the function is
increasing, decreasing, concave up, and concave down.

37. 38.

39. 40.

41. 42.

\hypertarget{section-1.4-composition-of-functions}{\subsection{Section
1.4 Composition of
Functions}\label{section-1.4-composition-of-functions}}

Suppose we wanted to calculate how much it costs to heat a house on a
particular day of the year. The cost to heat a house will depend on the
average daily temperature, and the average daily temperature depends on
the particular day of the year. Notice how we have just defined two
relationships: The temperature depends on the day, and the cost depends
on the temperature. Using descriptive variables, we can notate these two
functions.

The first function, $C(T),} gives the cost $C} of heating a
house when the average daily temperature is $T} degrees Celsius,
and the second, $T(d)}, gives the average daily temperature on day
$d$ of the year in some city. If we wanted to determine the cost of
heating the house on the 5\textsuperscript{th} day of the year, we could
do this by linking our two functions together, an idea called
composition of functions. Using the function $T(d)}, we could
evaluate $T}(5) to determine the average daily temperature on the
5\textsuperscript{th} day of the year. We could then use that
temperature as the input to the $C(T)} function to find the cost to
heat the house on the 5\textsuperscript{th} day of the year:
$C(T(5)).}

Composition of Functions

When the output of one function is used as the input of another, we call
the entire operation a \textbf{composition of functions}. We write
$f(g(x))}, and read this as ``$f$ of $g} of $x}'' or
``$f$ composed with $g} at $x}''.

An alternate notation for composition uses the composition operator:

is read ``$f$ of $g} of $x}'' or ``$f$ composed with
$g} at $x}'', just like $f(g(x)).}

Example 1

Suppose $c(s)} gives the number of calories burned doing $s}
sit-ups, and $s(t)} gives the number of sit-ups a person can do in
$t} minutes. Interpret $c}($s}(3))$.}

When we are asked to interpret, we are being asked to explain the
meaning of the expression in words. The inside expression in the
composition is $s}(3). Since the input to the $s} function is
time, the 3 is representing 3 minutes, and $s}(3) is the number of
sit-ups that can be done in 3 minutes. Taking this output and using it
as the input to the $c(s)} function will gives us the calories that
can be burned by the number of sit-ups that can be done in 3 minutes.

Note that it is not important that the same variable be used for the
output of the inside function and the input to the outside function.
However, it $is} essential that the units on the output of the
inside function match the units on the input to the outside function, if
the units are specified.

Example 2

Suppose $f(x)} gives miles that can be driven in $x} hours,
and $g(y)} gives the gallons of gas used in driving $y$ miles.
Which of these expressions is meaningful: $f(g(y))} or
$g(f(x))}?

The expression $g(y)} takes miles as the input and outputs a number
of gallons. The function $f(x)} is expecting a number of hours as
the input; trying to give it a number of gallons as input does not make
sense. Remember the units must match, and number of gallons does not
match number of hours, so the expression $f(g(y))} is meaningless.

The expression $f(x)} takes hours as input and outputs a number of
miles driven. The function $g(y)} is expecting a number of miles as
the input, so giving the output of the $f(x)} function (miles
driven) as an input value for g($y$), where gallons of gas depends
on miles driven, does make sense. The expression $g(f(x))} makes
sense, and will give the number of gallons of gas used, $g},
driving a certain number of miles, $f(x)}, in $x} hours.

Try it Now

1. In a department store you see a sign that says 50\% off clearance
merchandise, so final cost $C} depends on the clearance price,
$p}, according to the function $C(p)}. Clearance price,
$p}, depends on the original discount, $d$, given to the
clearance item, $p(d).} Interpret $C(p(d))}.

\textbf{Composition of Functions using Tables and Graphs}

When working with functions given as tables and graphs, we can look up
values for the functions using a provided table or graph, as discussed
in section 1.1. We start evaluation from the provided input, and first
evaluate the inside function. We can then use the output of the inside
function as the input to the outside function. To remember this, always
work from the inside out.

Example 3

Using the tables below, evaluateand

To evaluate, we start from the inside with the value 3. We then evaluate
the inside expressionusing the table that defines the function $g}:
.

We can then use that result as the input to the $f$ function, so is
replaced by the equivalent value 2 and we can evaluate. Then using the
table that defines the function $f$, we find that.

.

To evaluate , we first evaluate the inside expression using the first
table: $.} Then using the table for $g} we can evaluate:

.

Try it Now

2. Using the tables from the example above, evaluate and $.}

Example 4

Using the graphs below, evaluate $.}

\includegraphics[width=2.13770in,height=2.20000in]{media/image386.png}
\includegraphics[width=2.18990in,height=2.20000in]{media/image387.png}

To evaluate , we again start with the inside evaluation. We evaluate
using the graph of the $g(x)} function, finding the input of 1 on
the horizontal axis and finding the output value of the graph at that
input. Here, .

Using this value as the input to the $f$ function, . We can then
evaluate this by looking to the graph of the $f(x)} function,
finding the input of 3 on the horizontal axis, and reading the output
value of the graph at this input.

, so.

Try it Now

3. Using the graphs from the previous example, evaluate $.}

\textbf{Composition using Formulas}

When evaluating a composition of functions where we have either created
or been given formulas, the concept of working from the inside out
remains the same. First, we evaluate the inside function using the input
value provided, then use the resulting output as the input to the
outside function.

Example 5

Given and , evaluate .

Since the inside evaluation iswe start by evaluating the $h(x)}
function at 1:

Then , so we evaluate the $f(t)} function at an input of 5:

Try it Now

4. Using the functions from the example above, evaluate .

While we can compose the functions as above for each individual input
value, sometimes it would be really helpful to find a single formula
which will calculate the result of a composition $f(g(x))}. To do
this, we will extend our idea of function evaluation. Recall that when
we evaluate a function like, we put whatever value is inside the
parentheses after the function name into the formula wherever we see the
input variable.

Example 6

Given, evaluate and .

We could simplify the results above if we wanted to

We are not limited, however, to using a numerical value as the input to
the function. We can put anything into the function: a value, a
different variable, or even an algebraic expression, provided we use the
input expression everywhere we see the input variable.

Example 7

Using the function from the previous example, evaluate $f(a)}.

This means that the input value for $t} is some unknown quantity
$a$. As before, we evaluate by replacing the input variable
$t} with the input quantity, in this case $a$.

The same idea can then be applied to expressions more complicated than a
single letter.

Example 8

Using the same $f(t)} function from above, evaluate .

Everywhere in the formula for $f$ where there was a $t}, we
would replace it with the input . Since in the original formula the
input $t} was squared in the first term, the entire input needs to
be squared when we substitute, so we need to use grouping
parentheses$.} To avoid problems, it is advisable to always use
parentheses around inputs.

We could simplify this expression further to if we wanted to:

Use the ``FOIL'' technique (first, outside, inside, last)

distribute the negative sign combine like terms

Example 9

Using the same function, evaluate .

Note that in this example, the same variable is used in the input
expression and as the input variable of the function. This doesn't
matter -- we still replace the original input $t} in the formula
with the new input expression, .

Try it Now

5. Given , evaluate .

This now allows us to find an expression for a composition of functions.
If we want to find a formula for $f(g(x))}, we can start by writing
out the formula for $g(x)}. We can then evaluate the function
$f(x)} at that expression, as in the examples above.

Example 10

Let and , find $f(g(x))} and $g(f(x)).}

To find $f(g(x))}, we start by evaluating the inside, writing out
the formula for $g(x)}.

We then use the expression as input for the function $f$.

We then evaluate the function $f(x)} using the formula for
$g(x)} as the input.

Since ,

This gives us the formula for the composition: .

Likewise, to find $g(f(x))}, we evaluate the inside, writing out
the formula for $f(x)}

Now we evaluate the function $g(x)} using
$x}\textsuperscript{2} as the input.

Try it Now

6. Let and , find $f(g(x))} and $g(f(x)).}

Example 11

A city manager determines that the tax revenue, $R}, in millions of
dollars collected on a population of $p} thousand people is given
by the formula , and that the city's population, in thousands, is
predicted to follow the formula , where $t} is measured in years
after 2010. Find a formula for the tax revenue as a function of the
year.

Since we want tax revenue as a function of the year, we want year to be
our initial input, and revenue to be our final output. To find revenue,
we will first have to predict the city population, and then use that
result as the input to the tax function. So we need to find
$R(p(t))}. Evaluating this,

This composition gives us a single formula which can be used to predict
the tax revenue during a given year, without needing to find the
intermediary population value.

For example, to predict the tax revenue in 2017, when $t} = 7
(because $t} is measured in years after 2010),

million dollars

\textbf{Domain of Compositions}

When we think about the domain of a composition , we must consider both
the domain of the inner function and the domain of the composition
itself. While it is tempting to only look at the resulting composite
function, if the inner function were undefined at a value of $x},
the composition would not be possible.

Example 12

Let and . Find the domain of .

Since we want to avoid the square root of negative numbers, the domain
of is the set of values where . The domain is .

The composition is .

The composition is undefined when $x} = 3, so that value must also
be excluded from the domain. Notice that the composition doesn't involve
a square root, but we still have to consider the domain limitation from
the inside function.

Combining the two restrictions, the domain is all values of $x}
greater than or equal to 2, except $x} = 3.

In inequalities, the domain is: .

In interval notation, the domain is: .

Try it Now

7. Let and . Find the domain of .

\textbf{Decomposing Functions}

In some cases, it is desirable to decompose a function -- to write it as
a composition of two simpler functions.

Example 13

Write as the composition of two functions.

We are looking for two functions, $g} and $h$, so . To do
this, we look for a function inside a function in the formula for
$f(x)}. As one possibility, we might notice that is the inside of
the square root. We could then decompose the function as:

We can check our answer by recomposing the functions:

Note that this is not the only solution to the problem. Another
non-trivial decomposition would be and

Important Topics of this Section

Definition of Composition of Functions

Compositions using:

Words

Tables

Graphs

Equations

Domain of Compositions

Decomposition of Functions

Try it Now Answers

\begin{enumerate}
\def\labelenumi{\arabic{enumi}.}
\item
  \begin{quote}
  The final cost, $C}, depends on the clearance price, $p},
  which is based on the original discount, $d$. (Or the original
  discount $d$, determines the clearance price and the final cost
  is half of the clearance price.)
  \end{quote}
\item
  \begin{quote}
  and
  \end{quote}
\item
\item
  \begin{quote}
  $did you remember to insert your input values using
  parentheses?\\
  }
  \end{quote}
\item
\item
\item
  \begin{quote}
  is undefined at $x} = 0.\\
  The composition, is undefined when , when .\\
  Restricting these two values, the domain is .
  \end{quote}
\end{enumerate}

\subsection{Section 1.4 Exercises}\label{section-1.4-exercises}

Given each pair of functions, calculate and .

1. , 2. ,

3. , 4. ,

\includegraphics{media/image479.wmf}Use the table of values to evaluate
each expression

\begin{enumerate}
\def\labelenumi{\arabic{enumi}.}
\setcounter{enumi}{4}
\item
\item
\item
\item
\item
\item
\item
\item
\end{enumerate}

\includegraphics[width=2.07639in,height=2.11875in]{media/image489.png}Use
the graphs to evaluate the expressions below.

\begin{enumerate}
\def\labelenumi{\arabic{enumi}.}
\setcounter{enumi}{4}
\item
  \includegraphics[width=2.03819in,height=2.07361in]{media/image490.png}
\item
\item
\item
\item
\item
\item
\item
\end{enumerate}

\begin{quote}
For each pair of functions, find and . Simplify your answers.
\end{quote}

21. , 22. ,

23. , 24. ,

25. , 26. ,

\begin{enumerate}
\def\labelenumi{\arabic{enumi}.}
\setcounter{enumi}{26}
\item
  If ,and , find ~
\item
  If , and , find ~
\item
  The function gives the number of items that will be demanded when the
  price is $p}. The production cost, is the cost of producing
  $x} items. To determine the cost of production when the price is
  \$6, you would do which of the following:
\end{enumerate}

a. Evaluate b. Evaluate

c. Solve d. Solve

\begin{enumerate}
\def\labelenumi{\arabic{enumi}.}
\setcounter{enumi}{26}
\item
  The function gives the pain level on a scale of 0-10 experienced by a
  patient with $d$ milligrams of a pain reduction drug in their
  system. The milligrams of drug in the patient's system after $t}
  minutes is modeled by . To determine when the patient will be at a
  pain level of 4, you would need to:
\end{enumerate}

a. Evaluate b. Evaluate

c. Solve d. Solve

\begin{enumerate}
\def\labelenumi{\arabic{enumi}.}
\setcounter{enumi}{26}
\item
  The radius $r}, in inches, of a spherical balloon is related to
  the volume, $V}, by . Air is pumped into the balloon, so the
  volume after $t} seconds is given by .

  \begin{enumerate}
  \def\labelenumii{\alph{enumii}.}
  \item
    Find the composite function
  \item
    Find the radius after 20 seconds
  \end{enumerate}
\item
  The number of bacteria in a refrigerated food product is given by , ,
  where $T} is the temperature of the food. When the food is
  removed from the refrigerator, the temperature is given by , where
  $t} is the time in hours.
\end{enumerate}

\begin{enumerate}
\def\labelenumi{\alph{enumi}.}
\item
  Find the composite function
\item
  Find the bacteria count after 4 hours
\end{enumerate}

\begin{enumerate}
\def\labelenumi{\arabic{enumi}.}
\setcounter{enumi}{26}
\item
  Given and , find the domain of .
\end{enumerate}

34. Given and , find the domain of .

35. Given and , find the domain of .

36. Given and , find the domain of .

37. Given and , find the domain of .

38. Given and , find the domain of .

Find functions and so the given function can be expressed as.

39. 40.

41. 42.

43. 44.

45. Let be a linear function, with form for constants $a$ and
$b}. {[}UW{]}

\begin{enumerate}
\def\labelenumi{\alph{enumi}.}
\item
  Show that is a linear function
\item
  Find a function such that
\end{enumerate}

46. Let {[}UW{]}

\begin{enumerate}
\def\labelenumi{\alph{enumi}.}
\item
  Sketch the graphs of on the interval −2 ≤ $x} ≤ 10.
\item
  Your graphs should all intersect at the point (6, 6). The value x = 6
  is called a fixed point of the function $f(x)}since ; that is, 6
  is fixed - it doesn't move when $f$ is applied to it. Give an
  explanation for why 6 is a fixed point for any function .
\item
  Linear functions (with the exception of ) can have at most one fixed
  point. Quadratic functions can have at most two. Find the fixed points
  of the function .
\item
  Give a quadratic function whose fixed points are $x} = −2 and
  $x} = 3.
\end{enumerate}

47. A car leaves Seattle heading east. The speed of the car in mph after
$m} minutes is given by the function . {[}UW{]}

\begin{enumerate}
\def\labelenumi{\alph{enumi}.}
\item
  Find a function that converts seconds $s} into minutes $m}.
  Write out the formula for the new function ; what does this function
  calculate?
\item
  Find a function ) that converts hours $h$ into minutes $m}.
  Write out the formula for the new function ; what does this function
  calculate?
\item
  Find a function that converts mph $s} into ft/sec $z}. Write
  out the formula for the new function ; what does this function
  calculate?
\end{enumerate}

\hypertarget{section-1.5-transformation-of-functions}{\subsection{Section
1.5 Transformation of
Functions}\label{section-1.5-transformation-of-functions}}

Often when given a problem, we try to model the scenario using
mathematics in the form of words, tables, graphs and equations in order
to explain or solve it. When building models, it is often helpful to
build off of existing formulas or models. Knowing the basic graphs of
your tool-kit functions can help you solve problems by being able to
model new behavior by adapting something you already know.
Unfortunately, the models and existing formulas we know are not always
exactly the same as the ones presented in the problems we face.

Fortunately, there are systematic ways to shift, stretch, compress, flip
and combine functions to help them become better models for the problems
we are trying to solve. We can transform what we already know into what
we need, hence the name, ``Transformation of functions.'' When we have a
story problem, formula, graph, or table, we can then transform that
function in a variety of ways to form new functions.

\textbf{Shifts}

Example 1

To regulate temperature in a green building, air flow vents near the
roof open and close throughout the day to allow warm air to escape. The
graph below shows the open vents $V} (in square feet) throughout
the day, $t} in hours after midnight. During the summer, the
facilities staff decides to try to better regulate temperature by
increasing the amount of open vents by 20 square feet throughout the
day. Sketch a graph of this new function.

\includegraphics[width=2.69159in,height=2.20000in]{media/image582.png}\includegraphics[width=2.68029in,height=2.20000in]{media/image583.png}

We can sketch a graph of this new function by adding 20 to each of the
output values of the original function. This will have the effect of
shifting the graph up.

Notice that in the second graph, for each input value, the output value
has increased by twenty, so if we call the new function $S(t)}, we
could write .

Note that this notation tells us that for any value of $t},
$S(t)} can be found by evaluating the $V} function at the same
input, then adding twenty to the result.

This defines $S} as a transformation of the function $V}, in
this case a vertical shift up 20 units.

Notice that with a vertical shift the input values stay the same and
only the output values change.

Vertical Shift

Given a function $f(x)}, if we define a new function $g(x)} as

, where $k} is a constant

then $g(x)} is a \textbf{vertical shift} of the function
$f(x)}, where all the output values have been increased by
$k}.

If $k} is positive, then the graph will shift up

If $k} is negative, then the graph will shift down

Example 2

A function $f(x)} is given as a table below. Create a table for the
function

The formula tells us that we can find the output values of the $g}
function by subtracting 3 from the output values of the $f$
function. For example,

is found from the given table

is our given transformation

Subtracting 3 from each $f(x)} value, we can complete a table of
values for $g(x)}

As with the earlier vertical shift, notice the input values stay the
same and only the output values change.

Try it Now

1. The function gives the height $h$ of a ball (in meters) thrown
upwards from the ground after $t} seconds. Suppose the ball was
instead thrown from the top of a 10 meter building. Relate this new
height function $b(t)} to $h(t)}, then find a formula for
$b(t).}

The vertical shift is a change to the output, or outside, of the
function. We will now look at how changes to input, on the inside of the
function, change its graph and meaning.

Example 3

Returning to our building air flow example from the beginning of the
section, suppose that in Fall, the facilities staff decides that the
original venting plan starts too late, and they want to move the entire
venting program to start two hours earlier. Sketch a graph of the new
function.

\includegraphics[width=2.55846in,height=2.10000in]{media/image583.png}
\includegraphics[width=2.43479in,height=2.10000in]{media/image590.png}

$V(t)} = the original venting plan $F(t)} = starting 2 hours
sooner

In the new graph, which we can call $F(t)}, at each time, the air
flow is the same as the original function $V(t)} was two hours
later. For example, in the original function $V}, the air flow
starts to change at 8am, while for the function $F(t)} the air flow
starts to change at 6am. The comparable function values are .

Notice also that the vents first opened to 220 sq. ft. at 10 a.m. under
the original plan, while under the new plan the vents reach 220 sq. ft.
at 8 a.m., so .

In both cases we see that since $F(t)} starts 2 hours sooner, the
same output values are reached when,

Note that had the effect of shifting the graph to the $left}.

Horizontal changes or ``inside changes'' affect the domain of a function
(the input) instead of the range and often seem counterintuitive. The
new function $F(t)} uses the same outputs as $V(t)}, but
matches those outputs to inputs two hours earlier than those of
$V(t)}. Said another way, we must add 2 hours to the input of
$V} to find the corresponding output for $f$: .

Horizontal Shift

Given a function $f(x)}, if we define a new function $g(x)} as

, where $k} is a constant

then $g(x)} is a \textbf{horizontal shift} of the function
$f(x)}

If $k} is positive, then the graph will shift left

If $k} is negative, then the graph will shift right

Example 4

A function $f(x)} is given as a table below. Create a table for the
function

The formula tells us that the output values of $g} are the same as
the output value of $f$ with an input value three smaller. For
example, we know that. To get the same output from the $g}
function, we will need an input value that is 3 $larger}: We input
a value that is three larger for $g(x)} because the function takes
three away before evaluating the function $f$.

The result is that the function $g(x)} has been shifted to the
right by 3. Notice the output values for $g(x)} remain the same as
the output values for $f(x)} in the chart, but the corresponding
input values, $x}, have shifted to the right by 3: 2 shifted to 5,
4 shifted to 7, 6 shifted to 9 and 8 shifted to 11.

Example 5

\includegraphics[width=1.77847in,height=1.80139in]{media/image600.png}The
graph shown is a transformation of the toolkit function . Relate this
new function $g(x)} to $f(x)}, and then find a formula for
$g(x)}.

Notice that the graph looks almost identical in shape to the function,
but the $x} values are shifted to the right two units. The vertex
used to be at (0, 0) but now the vertex is at (2, 0) . The graph is the
basic quadratic function shifted two to the right, so

Notice how we must input the value $x} = 2, to get the output value
$y$ = 0; the $x} values must be two units larger, because of
the shift to the right by 2 units.

We can then use the definition of the $f(x)} function to write a
formula for $g(x)} by evaluating :

Since and

If you find yourself having trouble determining whether the shift is +2
or -2, it might help to consider a single point on the graph. For a
quadratic, looking at the bottom-most point is convenient. In the
original function, . In our shifted function, . To obtain the output
value of 0 from the $f$ function, we need to decide whether a +2 or
-2 will work to satisfy . For this to work, we will need to subtract 2
from our input values.

When thinking about horizontal and vertical shifts, it is good to keep
in mind that vertical shifts are affecting the output values of the
function, while horizontal shifts are affecting the input values of the
function.

Example 6

The function $G(m)} gives the number of gallons of gas required to
drive $m} miles. Interpret and .

is adding 10 to the output, gallons. This is 10 gallons of gas more than
is required to drive $m} miles. So, this is the gas required to
drive $m} miles, plus another 10 gallons of gas.

is adding 10 to the input, miles. This is the number of gallons of gas
required to drive 10 miles more than $m} miles.

Try it Now

2. Given the function graph the original function and the transformation
.\\
a. Is this a horizontal or a vertical change?\\
b. Which way is the graph shifted and by how many units?\\
c. Graph $f(x)} and $g(x)} on the same axes.

Now that we have two transformations, we can combine them together.

Remember:

Vertical Shifts are outside changes that affect the output (vertical)
axis values shifting the transformed function up or down.

Horizontal Shifts are inside changes that affect the input (horizontal)
axis values shifting the transformed function left or right.

Example 7

Given , sketch a graph of .

The function $f$ is our toolkit absolute value function. We know
that this graph has a V shape, with the point at the origin. The graph
of $h$ has transformed $f$ in two ways: is a change on the
inside of the function, giving a horizontal shift left by 1, then the
subtraction by 3 in is a change to the outside of the function, giving a
vertical shift down by 3. Transforming the graph gives

\includegraphics[width=2.47056in,height=2.45625in]{media/image617.png}

We could also find a formula for this transformation by evaluating the
expression for $h(x)}:

Example 8

\includegraphics[width=2.22014in,height=2.20278in]{media/image619.png}Write
a formula for the graph shown, a transformation of the toolkit square
root function.

The graph of the toolkit function starts at the origin, so this graph
has been shifted 1 to the right, and up 2. In function notation, we
could write that as . Using the formula for the square root function we
can write

Note that this transformation has changed the domain and range of the
function. This new graph has domain and range .

\textbf{Reflections}

Another transformation that can be applied to a function is a reflection
over the horizontal or vertical axis.

Example 9

Reflect the graph of both vertically and horizontally.

\includegraphics[width=1.99730in,height=2.00000in]{media/image625.png}Reflecting
the graph vertically, each output value will be reflected over the
horizontal $t} axis:

\includegraphics[width=1.94588in,height=2.00000in]{media/image626.png}

Since each output value is the opposite of the original output value, we
can write

Notice this is an outside change or vertical change that affects the
output $s(t)} values so the negative sign belongs outside of the
function.

Reflecting horizontally, each input value will be reflected over the
vertical axis.

\includegraphics[width=1.99028in,height=2.00000in]{media/image629.png}

Since each input value is the opposite of the original input value, we
can write

Notice this is an inside change or horizontal change that affects the
input values so the negative sign is on the inside of the function.

Note that these transformations can affect the domain and range of the
functions. While the original square root function has domain and range
, the vertical reflection gives the $V(t)} function the range , and
the horizontal reflection gives the $H(t)} function the domain .

Reflections

Given a function $f(x)}, if we define a new function $g(x)} as

,

then $g(x)} is a \textbf{vertical reflection} of the function
$f(x)}, sometimes called a reflection about the $x}-axis

If we define a new function $g(x)} as

,

then $g(x)} is a \textbf{horizontal reflection} of the function
$f(x)}, sometimes called a reflection about the $y$-axis

Example 10

A function $f(x)} is given as a table below. Create a table for the
function and

For $g(x)}, this is a vertical reflection, so the $x} values
stay the same and each output value will be the opposite of the original
output value

For $h(x)}, this is a horizontal reflection, and each input value
will be the opposite of the original input value and the $h(x)}
values stay the same as the $f(x)} values:

Example 11

\includegraphics[width=2.21528in,height=1.45208in]{media/image638.png}A
common model for learning has an equation similar to , where $k} is
the percentage of mastery that can be achieved after $t} practice
sessions. This is a transformation of the function shown here. Sketch a
graph of $k(t)}.

This equation combines three transformations into one equation.

A horizontal reflection: combined with

A vertical reflection: combined with

A vertical shift up 1:

We can sketch a graph by applying these transformations one at a time to
the original function:

The original graph Horizontally reflected Then vertically reflected

\includegraphics[width=1.70000in,height=1.67062in]{media/image644.png}
\includegraphics[width=1.70000in,height=1.70169in]{media/image645.png}
\includegraphics[width=1.70000in,height=1.72164in]{media/image646.png}

\includegraphics[width=2.37500in,height=2.47153in]{media/image647.png}

Then, after shifting up 1, we get the final graph.

.

Note: As a model for learning, this function would be limited to a
domain of , with corresponding range .

Try it Now

3. Given the toolkit function , graph $g(x) = −f(x)} and $h(x)
= f(−x).}\\
Do you notice anything surprising?

Some functions exhibit symmetry, in which reflections result in the
original graph. For example, reflecting the toolkit functions or about
the $y$-axis will result in the original graph. We call these types
of graphs symmetric about the $y$-axis.

Likewise, if the graphs of or were reflected over both axes, the result
would be the original graph:

\includegraphics[width=1.77131in,height=1.80000in]{media/image657.png}
\includegraphics[width=1.81265in,height=1.80000in]{media/image658.png}
\includegraphics[width=1.74959in,height=1.80000in]{media/image659.png}

We call these graphs symmetric about the origin.

Even and Odd Functions

A function is called an \textbf{even function} if

The graph of an even function is symmetric about the vertical axis

A function is called an \textbf{odd function} if

The graph of an odd function is symmetric about the origin

Note: A function can be neither even nor odd if it does not exhibit
either symmetry. For example, the function is neither even nor odd.

Example 12

Is the function even, odd, or neither?

Without looking at a graph, we can determine this by finding formulas
for the reflections, and seeing if they return us to the original
function:

This does not return us to the original function, so this function is
not even.

We can now try also applying a horizontal reflection:

Since , this is an odd function.

\textbf{Stretches and Compressions}

With shifts, we saw the effect of adding or subtracting to the inputs or
outputs of a function. We now explore the effects of multiplying the
inputs or outputs.

Remember, we can transform the inside (input values) of a function or we
can transform the outside (output values) of a function. Each change has
a specific effect that can be seen graphically.

Example 13

\includegraphics[width=1.98898in,height=2.00000in]{media/image667.png}A
function $P(t)} models the growth of a population of fruit flies.
The growth is shown in the graph. A scientist is comparing this to
another population, $Q}, that grows the same way, but starts twice
as large. Sketch a graph of this population.

Since the population is always twice as large, the new population's
output values are always twice the original function output values.
Graphically, this would look like the second graph shown.

\includegraphics[width=1.96806in,height=2.00000in]{media/image668.png}Symbolically,

This means that for any input $t}, the value of the $Q}
function is twice the value of the $P} function. Notice the effect
on the graph is a vertical stretching of the graph, where every point
doubles its distance from the horizontal axis. The input values,
$t}, stay the same while the output values are twice as large as
before.

Vertical Stretch/Compression

Given a function $f(x)}, if we define a new function $g(x)} as

, where $k} is a constant

then $g(x)} is a \textbf{vertical stretch or compression} of the
function $f(x).}

If $k} \textgreater{} 1, then the graph will be stretched

If 0\textless{} $k} \textless{} 1, then the graph will be
compressed

If $k} \textless{} 0, then there will be combination of a vertical
stretch or compression with a vertical reflection

Example 14

A function $f(x)} is given as a table below. Create a table for the
function

The formula tells us that the output values of $g} are half of the
output values of $f$ with the same inputs. For example, we know
that. Then

The result is that the function $g(x)} has been compressed
vertically by ½. Each output value has been cut in half, so the graph
would now be half the original height.

Example 15

\includegraphics[width=2.38819in,height=2.47361in]{media/image675.png}The
graph shown is a transformation of the toolkit function . Relate this
new function $g(x)} to $f(x)}, then find a formula for
$g(x)}.

When trying to determine a vertical stretch or shift, it is helpful to
look for a point on the graph that is relatively clear. In this graph,
it appears that . With the basic cubic function at the same input, .

Based on that, it appears that the outputs of $g} are ¼ the outputs
of the function $f$, since .

From this we can fairly safely conclude that:

We can write a formula for $g} by using the definition of the
function $f$

Now we consider changes to the inside of a function.

Example 16

Returning to the fruit fly population we looked at earlier, suppose the
scientist is now comparing it to a population that progresses through
its lifespan twice as fast as the original population. In other words,
this new population, $R}, will progress in 1 hour the same amount
the original population did in 2 hours, and in 2 hours, will progress as
much as the original population did in 4 hours. Sketch a graph of this
population.

Symbolically, we could write

, and in general,

Graphing this,

Original population, $P(t)} Transformed, $R(t)}

\includegraphics[width=2.11806in,height=2.14861in]{media/image685.png}
\includegraphics[width=2.14583in,height=2.18750in]{media/image686.png}

Note the effect on the graph is a horizontal compression, where all
input values are half their original distance from the vertical axis.

Horizontal Stretch/Compression

Given a function $f(x)}, if we define a new function $g(x)} as

, where $k} is a constant

then $g(x)} is a \textbf{horizontal stretch or compression} of the
function $f(x).}

If $k} \textgreater{} 1, then the graph will be compressed by

If 0\textless{} $k} \textless{} 1, then the graph will be stretched
by

If $k} \textless{} 0, then there will be combination of a
horizontal stretch or compression with a horizontal reflection.

Example 17

A function $f(x)} is given as a table below. Create a table for the
function

The formula tells us that the output values for $g} are the same as
the output values for the function $f$ at an input half the size.
Notice that we don't have enough information to determine since , and we
do not have a value for in our table. Our input values to $g} will
need to be twice as large to get inputs for $f$ that we can
evaluate. For example, we can determine since .

Since each input value has been doubled, the result is that the function
$g(x)} has been stretched horizontally by 2.

Example 18

Two graphs are shown below. Relate the function $g(x)} to
$f(x)}.

\includegraphics[width=2.70000in,height=1.95052in]{media/image697.png}
\includegraphics[width=2.70000in,height=1.92768in]{media/image698.png}

The graph of $g(x)} looks like the graph of $f(x)}
horizontally compressed. Since $f(x)} ends at (6,4) and $g(x)}
ends at (2,4) we can see that the $x} values have been compressed
by 1/3, because 6(1/3) = 2. We might also notice that , and . Either
way, we can describe this relationship as . This is a horizontal
compression by 1/3.

Notice that the coefficient needed for a horizontal stretch or
compression is the $reciprocal} of the stretch or compression. To
stretch the graph horizontally by 4, we need a coefficient of 1/4 in our
function: . This means the input values must be four times larger to
produce the same result, requiring the input to be larger, causing the
horizontal stretching.

Try it Now

4. Write a formula for the toolkit square root function horizontally
stretched by three.

It is useful to note that for most toolkit functions, a horizontal
stretch or vertical stretch can be represented in other ways. For
example, a horizontal compression of the function by ½ would result in a
new function , but this can also be written as , a vertical stretch of
$f(x)} by 4. When writing a formula for a transformed toolkit, we
only need to find one transformation that would produce the graph.

\textbf{Combining Transformations}

When combining transformations, it is very important to consider the
order of the transformations. For example, vertically shifting by 3 and
then vertically stretching by 2 does not create the same graph as
vertically stretching by 2 then vertically shifting by 3.

When we see an expression like , which transformation should we start
with? The answer here follows nicely from order of operations, for
outside transformations. Given the output value of $f(x)}, we first
multiply by 2, causing the vertical stretch, then add 3, causing the
vertical shift. (Multiplication before Addition)

Combining Vertical Transformations

When combining vertical transformations written in the form ,

first vertically stretch by $a$, then vertically shift by $k.}

Horizontal transformations are a little trickier to think about. When we
write for example, we have to think about how the inputs to the $g}
function relate to the inputs to the $f$ function. Suppose we know
. What input to $g} would produce that output? In other words, what
value of $x} will allow ? We would need . To solve for $x}, we
would first subtract 3, resulting in horizontal shift, then divide by 2,
causing a horizontal compression.

Combining Horizontal Transformations

When \textbf{combining horizontal transformations} written in the form,

first horizontally shift by $p}, then horizontally stretch by
1/$b.}

This format ends up being very difficult to work with, since it is
usually much easier to horizontally stretch a graph before shifting. We
can work around this by factoring inside the function.

=

Factoring in this way allows us to horizontally stretch first, then
shift horizontally.

Combining Horizontal Transformations (Factored Form)

When \textbf{combining horizontal transformations} written in the form ,

first horizontally stretch by 1/$b}, then horizontally shift by
$h$.

Independence of Horizontal and Vertical Transformations

\textbf{Horizontal and vertical transformations are independent}. It
does not matter whether horizontal or vertical transformations are done
first.

Example 19

Given the table of values for the function $f(x)} below, create a
table of values for the function

There are 3 steps to this transformation and we will work from the
inside out. Starting with the horizontal transformations, is a
horizontal compression by 1/3, which means we multiply each $x}
value by 1/3.

\includegraphics{media/image717.wmf}

Looking now to the vertical transformations, we start with the vertical
stretch, which will multiply the output values by 2. We apply this to
the previous transformation.

\includegraphics{media/image718.wmf}

Finally, we can apply the vertical shift, which will add 1 to all the
output values.

\includegraphics{media/image719.wmf}

Example 20

Using the graph of $f(x)} below, sketch a graph of

\includegraphics[width=2.56458in,height=2.56181in]{media/image721.png}

To make things simpler, we'll start by factoring out the inside of the
function

By factoring the inside, we can first horizontally stretch by 2, as
indicated by the ½ on the inside of the function. Remember twice the
size of 0 is still 0, so the point (0,2) remains at (0,2) while the
point (2,0) will stretch to (4,0).

Next, we horizontally shift left by 2 units, as indicated by the
$x}+2.

Last, we vertically shift down by 3 to complete our sketch, as indicated
by the -3 on the outside of the function.

Horizontal stretch by 2 Horizontal shift left by 2 Vertical shift down 3

\includegraphics[width=1.80000in,height=1.77527in]{media/image723.png}
\includegraphics[width=1.80000in,height=1.76864in]{media/image724.png}
\includegraphics[width=1.80000in,height=1.76617in]{media/image725.png}

Example 21

Write an equation for the transformed graph of the quadratic function
shown.

\includegraphics[width=2.10208in,height=2.17986in]{media/image726.png}

Since this is a quadratic function, first consider what the basic
quadratic tool kit function looks like and how this has changed.
Observing the graph, we notice several transformations:

The original tool kit function has been flipped over the $x} axis,
some kind of stretch or compression has occurred, and we can see a shift
to the right 3 units and a shift up 1 unit.

In total there are four operations:

Vertical reflection, requiring a negative sign outside the function

Vertical Stretch $or} Horizontal Compression\textsuperscript{*}

Horizontal Shift Right 3 units, which tells us to put $x}-3 on the
inside of the function

Vertical Shift up 1 unit, telling us to add 1 on the outside of the
function

\textsuperscript{*} It is unclear from the graph whether it is showing a
vertical stretch or a horizontal compression. For the quadratic, it
turns out we could represent it either way, so we'll use a vertical
stretch. You may be able to determine the vertical stretch by
observation.

By observation, the basic tool kit function has a vertex at (0, 0) and
symmetrical points at (1, 1) and (-1, 1). These points are one unit up
and one unit over from the vertex. The new points on the transformed
graph are one unit away horizontally but 2 units away vertically. They
have been stretched vertically by two.

Not everyone can see this by simply looking at the graph. If you can,
great, but if not, we can solve for it. First, we will write the
equation for this graph, with an unknown vertical stretch.

The original function

Vertically reflected

Vertically stretched

Shifted right 3

Shifted up 1

We now know our graph is going to have an equation of the form . To find
the vertical stretch, we can identify any point on the graph (other than
the highest point), such as the point (2,-1), which tells us . Using our
general formula, and substituting 2 for $x}, and -1 for $g(x)}

This tells us that to produce the graph we need a vertical stretch by
two.

The function that produces this graph is therefore .

Try it Now

5. Consider the linear function . Describe its transformation in words
using the identity tool kit function $f(x)} = $x} as a
reference.

Example 22

On what interval(s) is the function increasing and decreasing?

This is a transformation of the toolkit reciprocal squared function, :

A vertical flip and vertical stretch by 2

A shift right by 1

A shift up by 3

\includegraphics[width=1.92639in,height=1.98542in]{media/image742.png}The
basic reciprocal squared function is increasing on and decreasing on .
Because of the vertical flip, the $g(x)} function will be
decreasing on the left and increasing on the right. The horizontal shift
right by 1 will also shift these intervals to the right one. From this,
we can determine $g(x)} will be increasing on and decreasing on .
We also could graph the transformation to help us determine these
intervals.

Try it Now

6. On what interval(s) is the function concave up and down?

Important Topics of This Section

Transformations

Vertical Shift (up \& down)

Horizontal Shifts (left \& right)

Reflections over the vertical \& horizontal axis

Even \& Odd functions

Vertical Stretches \& Compressions

Horizontal Stretches \& Compressions

Combinations of Transformation

Try it Now Answers

\includegraphics[width=2.01667in,height=1.31042in]{media/image748.png}1.

2. a. Horizontal shift

b. The function is shifted to the LEFT by 2 units.

c. Shown to the right

\includegraphics[width=1.84028in,height=1.86458in]{media/image750.png}

3. Shown to the right

Notice: $g(x) = f(-x)} looks the same as $f(x)}

4. so using the square root function we get

5. The identity tool kit function $f(x) = x} has been transformed
in 3 steps

a. Vertically stretched by 2.

b. Vertically reflected over the $x} axis.

c. Vertically shifted up by 1 unit.

6. $h(t)} is concave down on and concave up on

\subsection{Section 1.5 Exercises}\label{section-1.5-exercises}

Describe how each function is a transformation of the original function

1. 2.

3. 4.

5. 6.

7. 8.

9. 10.

\begin{enumerate}
\def\labelenumi{\arabic{enumi}.}
\setcounter{enumi}{10}
\item
  Write a formula for shifted up 1 unit and left 2 units.
\item
  Write a formula for shifted down 3 units and right 1 unit.
\item
  Write a formula for shifted down 4 units and right 3 units.
\item
  Write a formula for shifted up 2 units and left 4 units.
\item
  Tables of values for , , and are given below. Write and as
  transformations of .
\end{enumerate}

\begin{longtable}[]{@{}llllllll@{}}
\toprule
$\textbf{x}} & $\textbf{f(x)}} & & $\textbf{x}} &
$\textbf{g(x)}} & & $\textbf{x}} &
$\textbf{h(x)}}\tabularnewline
\midrule
\endhead
-2 & -2 & & -1 & -2 & & -2 & -1\tabularnewline
-1 & -1 & & 0 & -1 & & -1 & 0\tabularnewline
0 & -3 & & 1 & -3 & & 0 & -2\tabularnewline
1 & 1 & & 2 & 1 & & 1 & 2\tabularnewline
2 & 2 & & 3 & 2 & & 2 & 3\tabularnewline
\bottomrule
\end{longtable}

\begin{enumerate}
\def\labelenumi{\arabic{enumi}.}
\setcounter{enumi}{10}
\item
  Tables of values for , , and are given below. Write and as
  transformations of .
\end{enumerate}

\begin{longtable}[]{@{}llllllll@{}}
\toprule
$\textbf{x}} & $\textbf{f(x)}} & & $\textbf{x}} &
$\textbf{g(x)}} & & $\textbf{x}} &
$\textbf{h(x)}}\tabularnewline
\midrule
\endhead
-2 & -1 & & -3 & -1 & & -2 & -2\tabularnewline
-1 & -3 & & -2 & -3 & & -1 & -4\tabularnewline
0 & 4 & & -1 & 4 & & 0 & 3\tabularnewline
1 & 2 & & 0 & 2 & & 1 & 1\tabularnewline
2 & 1 & & 1 & 1 & & 2 & 0\tabularnewline
\bottomrule
\end{longtable}

\includegraphics[width=1.77708in,height=1.79097in]{media/image782.png}The
graph of is shown. Sketch a graph of each transformation of .

\begin{enumerate}
\def\labelenumi{\arabic{enumi}.}
\setcounter{enumi}{16}
\item
\item
\item
\item
\end{enumerate}

Sketch a graph of each function as a transformation of a toolkit
function.

\begin{enumerate}
\def\labelenumi{\arabic{enumi}.}
\setcounter{enumi}{20}
\item
\item
\item
\item
\end{enumerate}

~Write an equation for each function graphed below.

25.\includegraphics[width=1.87500in,height=1.86196in]{media/image793.png}
26.
\includegraphics[width=1.84722in,height=1.84936in]{media/image794.png}

27.
\includegraphics[width=1.90278in,height=1.87195in]{media/image795.png}
28.\includegraphics[width=1.86111in,height=1.86327in]{media/image796.png}

Find a formula for each of the transformations of the square root whose
graphs are given below.\\
29.
\includegraphics[width=1.83659in,height=1.86000in]{media/image797.png}
30.
\includegraphics[width=1.90187in,height=1.86000in]{media/image798.png}

~

~

\includegraphics[width=1.78446in,height=1.80000in]{media/image782.png}The
graph of is shown. Sketch a graph of each transformation of

\begin{enumerate}
\def\labelenumi{\arabic{enumi}.}
\setcounter{enumi}{30}
\item
\item
\item
  Starting with the graph of write the equation of the graph that
  results from\\
  a. reflecting about the $x}-axis and the $y$-axis

  b. reflecting about the ­$x}-axis, shifting left 2 units, and
  down 3 units
\end{enumerate}

~

\begin{enumerate}
\def\labelenumi{\arabic{enumi}.}
\setcounter{enumi}{30}
\item
  Starting with the graph of write the equation of the graph that
  results from\\
  a. reflecting about the $x}-axis

  b. reflecting about the ­$y$-axis, shifting right 4 units, and up
  2 units
\end{enumerate}

Write an equation for each function graphed below.

35.
\includegraphics[width=1.79861in,height=1.84857in]{media/image809.png}
36.
\includegraphics[width=1.82463in,height=1.85000in]{media/image810.png}

37.
\includegraphics[width=1.86513in,height=1.85000in]{media/image811.png}
38.\includegraphics[width=1.85645in,height=1.85000in]{media/image812.png}

39. For each equation below, determine if the function is Odd, Even, or
Neither.

\begin{enumerate}
\def\labelenumi{\alph{enumi}.}
\item
\item
\item
\end{enumerate}

~

40. For each equation below, determine if the function is Odd, Even, or
Neither.

\begin{enumerate}
\def\labelenumi{\alph{enumi}.}
\item
\item
\item
\end{enumerate}

~

Describe how each function is a transformation of the original function
.

41. 42.

43. 44.

45. 46.

47. 48.

49. 50.

Write a formula for the function that results when the given toolkit
function is transformed as described.

\begin{enumerate}
\def\labelenumi{\arabic{enumi}.}
\setcounter{enumi}{50}
\item
  reflected over the $y$ axis and horizontally compressed by a
  factor of .
\item
  reflected over the $x} axis and horizontally stretched by a
  factor of 2.
\item
  vertically compressed by a factor of , then shifted to the left 2
  units and down 3 units.
\item
  vertically stretched by a factor of 8, then shifted to the right 4
  units and up 2 units.
\item
  horizontally compressed by a factor of , then shifted to the right 5
  units and up 1 unit.
\item
  horizontally stretched by a factor of 3, then shifted to the left 4
  units and down 3 units.
\end{enumerate}

Describe how each formula is a transformation of a toolkit function.
Then sketch a graph of the transformation.

57. 58.

59. 60.

61. 62.

63. 64.

65. 66.

Determine the interval(s) on which the function is increasing and
decreasing.

67. 68.

69. 70.

Determine the interval(s) on which the function is concave up and
concave down.

71. 72.

73. 74.

\includegraphics[width=1.43889in,height=1.42847in]{media/image851.png}\includegraphics{media/image852.wmf}The
function is graphed here. Write an equation for each graph below as a
transformation of .

75.\includegraphics[width=1.77465in,height=1.75000in]{media/image855.png}76.\includegraphics[width=1.74395in,height=1.75000in]{media/image856.png}77.\includegraphics[width=1.79365in,height=1.75000in]{media/image857.png}

78.\includegraphics[width=1.76788in,height=1.75000in]{media/image858.png}79.
\includegraphics[width=1.76051in,height=1.75000in]{media/image859.png}80.
\includegraphics[width=1.74597in,height=1.75000in]{media/image860.png}

81.\includegraphics[width=1.75629in,height=1.75000in]{media/image861.png}82.
\includegraphics[width=1.74628in,height=1.75000in]{media/image862.png}83.
\includegraphics[width=1.74063in,height=1.75000in]{media/image863.png}

84.
\includegraphics[width=1.75000in,height=1.73978in]{media/image864.png}85.
\includegraphics[width=1.75000in,height=1.73947in]{media/image865.png}86.\includegraphics[width=1.75000in,height=1.74520in]{media/image866.png}

Write an equation for each transformed toolkit function graphed below.

87.
\includegraphics[width=1.75000in,height=1.75000in]{media/image867.png}88.
\includegraphics[width=1.75000in,height=1.73768in]{media/image868.png}89.
\includegraphics[width=1.75000in,height=1.75982in]{media/image869.png}

90.
\includegraphics[width=1.75000in,height=1.77827in]{media/image870.png}91.\includegraphics[width=1.75000in,height=1.76255in]{media/image871.png}92.\includegraphics[width=1.75000in,height=1.75000in]{media/image872.png}93.\includegraphics[width=1.75000in,height=1.76217in]{media/image873.png}94.\includegraphics[width=1.75000in,height=1.75608in]{media/image874.png}95.
\includegraphics[width=1.75000in,height=1.77681in]{media/image875.png}

96.\includegraphics[width=1.75000in,height=1.75998in]{media/image876.png}97.\includegraphics[width=1.75000in,height=1.72746in]{media/image877.png}98.\includegraphics[width=1.75000in,height=1.77498in]{media/image878.png}

Write a formula for the piecewise function graphed below.

99.
\includegraphics[width=2.55705in,height=1.80000in]{media/image879.png}
100.
\includegraphics[width=2.52919in,height=1.80000in]{media/image880.png}

101.
\includegraphics[width=2.17322in,height=2.20000in]{media/image881.png}
102.
\includegraphics[width=2.23834in,height=1.90000in]{media/image882.png}

103. Suppose you have a function such that the domain of is 1 ≤ $x}
≤ 6 and the range of is −3 ≤ $y$ ≤ 5. {[}UW{]}

\begin{enumerate}
\def\labelenumi{\alph{enumi}.}
\item
  What is the domain of?
\item
  What is the range of ?
\item
  What is the domain of ?
\item
  What is the range of ?
\item
  Can you find constants $B} and $C} so that the domain of is
  8 ≤ $x} ≤ 9?
\item
  Can you find constants $a$ and $d$ so that the range of is 0
  ≤ $y$ ≤ 1?
\end{enumerate}

\hypertarget{section-1.6-inverse-functions}{\subsection{Section 1.6
Inverse Functions}\label{section-1.6-inverse-functions}}

A fashion designer is travelling to Milan for a fashion show. He asks
his assistant, Betty, what 75 degrees Fahrenheit is in Celsius, and
after a quick search on Google, she finds the formula . Using this
formula, she calculates degrees Celsius. The next day, the designer
sends his assistant the week's weather forecast for Milan, and asks her
to convert the temperatures to Fahrenheit.

\includegraphics[width=3.21875in,height=0.89583in]{media/image894.png}

At first, Betty might consider using the formula she has already found
to do the conversions. After all, she knows her algebra, and can easily
solve the equation for $f$ after substituting a value for $C}.
For example, to convert 26 degrees Celsius, she could write:

After considering this option for a moment, she realizes that solving
the equation for each of the temperatures would get awfully tedious, and
realizes that since evaluation is easier than solving, it would be much
more convenient to have a different formula, one which takes the Celsius
temperature and outputs the Fahrenheit temperature. This is the idea of
an inverse function, where the input becomes the output and the output
becomes the input.

Inverse Function

If , then a function $g(x)} is an \textbf{inverse} of $f$ if .

The inverse of $f(x)} is typically notated, which is read
``$f$ inverse of $x}'', so equivalently, if then .

\textbf{Important:} The raised -1 used in the notation for inverse
functions is simply a notation, and does not designate an exponent or
power of -1.

Example 1

If for a particular function,, what do we know about the inverse?

The inverse function reverses which quantity is input and which quantity
is output, so if , then .

Alternatively, if you want to re-name the inverse function $g(x)},
then $g}(4) = 2

Try it Now

1. Given that , what do we know about the original function $h(x)}?

Notice that original function and the inverse function $undo} each
other. If , then , returning us to the original input. More simply put,
if you compose these functions together you get the original input as
your answer.

and

\includegraphics{media/image898.wmf}

Since the outputs of the function $f$ are the inputs to , the range
of $f$ is also the domain of . Likewise, since the inputs to
$f$ are the outputs of , the domain of $f$ is the range of .

Basically, like how the input and output values switch, the domain \&
ranges switch as well. But be careful, because sometimes a function
doesn't even have an inverse function, or only has an inverse on a
limited domain. For example, the inverse of is , since a square
``undoes'' a square root, but it is only the inverse of $f(x)} on
the domain {[}0,∞), since that is the range of .

Example 2

The function has domain and range , what would we expect the domain and
range of to be?

We would expect to swap the domain and range of $f$, so would have
domain and range .

Example 3

A function $f(t)} is given as a table below, showing distance in
miles that a car has traveled in $t} minutes. Find and interpret

The inverse function takes an output of $f$ and returns an input
for $f$. So in the expression, the 70 is an output value of the
original function, representing 70 miles. The inverse will return the
corresponding input of the original function $f$, 90 minutes, so.
Interpreting this, it means that to drive 70 miles, it took 90 minutes.

Alternatively, recall the definition of the inverse was that if then .
By this definition, if you are given then you are looking for a value
$a$ so that. In this case, we are looking for a $t} so that,
which is when $t} = 90.

Try it Now

2. Using the table below

Find and interpret the following

a.

b.

Example 4

A function $g(x)} is given as a graph below. Find and

\includegraphics[width=2.98542in,height=2.09861in]{media/image929.png}

To evaluate, we find 3 on the horizontal axis and find the corresponding
output value on the vertical axis. The point (3, 1) tells us that

To evaluate , recall that by definition means g($x}) = 3. By
looking for the output value 3 on the vertical axis we find the point
(5, 3) on the graph, which means g(5) = 3, so by definition.

Try it Now

3. Using the graph in Example 4 above

a. find

b. estimate

Example 5

Returning to our designer's assistant, find a formula for the inverse
function that gives Fahrenheit temperature given a Celsius temperature.

A quick Google search would find the inverse function, but
alternatively, Betty might look back at how she solved for the
Fahrenheit temperature for a specific Celsius value, and repeat the
process in general

By solving in general, we have uncovered the inverse function. If

Then

In this case, we introduced a function $h$ to represent the
conversion since the input and output variables are descriptive, and
writing could get confusing.

It is important to note that not all functions will have an inverse
function. Since the inverse takes an output of $f$ and returns an
input of $f$, in order for to itself be a function, then each
output of $f$ (input to ) must correspond to exactly one input of
$f$ (output of ) in order for to be a function. You might recall
that this is the definition of a one-to-one function.

Properties of Inverses

In order \textbf{for a function to have an inverse}, it must be a
one-to-one function.

In some cases, it is desirable to have an inverse for a function even
though the function is not one-to-one. In those cases, we can often
limit the domain of the original function to an interval on which the
function $is} one-to-one, then find an inverse only on that
interval.

If you have not already done so, go back to the toolkit functions that
were not one-to-one and limit or restrict the domain of the original
function so that it is one-to-one. If you are not sure how to do this,
proceed to Example 6.

Example 6

\includegraphics[width=1.90417in,height=1.91111in]{media/image940.png}The
quadratic function is not one-to-one. Find a domain on which this
function is one-to-one, and find the inverse on that domain.

We can limit the domain to to restrict the graph to a portion that is
one-to-one, and find an inverse on this limited domain.

You may have already guessed that since we undo a square with a square
root, the inverse of on this domain is .

You can also solve for the inverse function algebraically. If , we can
introduce the variable $y$ to represent the output values, allowing
us to write. To find the inverse we solve for the input variable

\includegraphics[width=2.02917in,height=2.02153in]{media/image945.png}To
solve for $x} we take the square root of each side. and get , so .
We have restricted $x} to being non-negative, so we'll use the
positive square root, or . In cases like this where the variables are
not descriptive, it is common to see the inverse function rewritten with
the variable $x}: . Rewriting the inverse using the variable
$x} is often required for graphing inverse functions using
calculators or computers.

Note that the domain and range of the square root function do correspond
with the range and domain of the quadratic function on the limited
domain. In fact, if we graph $h(x)} on the restricted domain and on
the same axes, we can notice symmetry: the graph of is the graph of
$h(x)} reflected over the line $y$ = $x}.

Example 7

\includegraphics[width=2.10000in,height=2.09444in]{media/image954.png}Given
the graph of $f(x)} shown, sketch a graph of .

This is a one-to-one function, so we will be able to sketch an inverse.
Note that the graph shown has an apparent domain of (0,∞) and range of
(-∞,∞), so the inverse will have a domain of (-∞,∞) and range of (0,∞).

Reflecting this graph of the line $y$ = $x}, the point (1, 0)
reflects to (0, 1), and the point (4, 2) reflects to (2, 4). Sketching
the inverse on the same axes as the original graph:

\includegraphics[width=2.15417in,height=2.17153in]{media/image956.png}

Important Topics of this Section

Definition of an inverse function

Composition of inverse functions yield the original input value

Not every function has an inverse function

To have an inverse a function must be one-to-one

Restricting the domain of functions that are not one-to-one.

Try it Now Answers

1.

2.a. . In 60 minutes, 50 miles are traveled.

b. . To travel 60 miles, it will take 70 minutes.

3. a.

b. (this is an approximation -- answers may vary slightly)

\subsection{Section 1.6 Exercises}\label{section-1.6-exercises}

Assume that the function $f$ is a one-to-one function.

1. If , find 2. If , find

3. If , find 4. If , find\\
5. If , find 6. If , find

\includegraphics[width=1.78125in,height=1.81319in]{media/image974.png}

7. Using the graph of shown

\begin{enumerate}
\def\labelenumi{\alph{enumi}.}
\item
  Find
\item
  Solve
\item
  Find
\item
  Solve
\end{enumerate}

~

\includegraphics[width=1.77986in,height=1.78750in]{media/image980.png}

8. Using the graph shown

\begin{enumerate}
\def\labelenumi{\alph{enumi}.}
\item
  Find
\item
  Solve
\item
  Find
\item
  Solve
\end{enumerate}

9. Use the table below to find the indicated quantities.

\begin{longtable}[]{@{}lllllllllll@{}}
\toprule
$\textbf{x}} & 0 & 1 & 2 & 3 & 4 & 5 & 6 & 7 & 8 & 9\tabularnewline
\midrule
\endhead
$\textbf{f(x)}} & 8 & 0 & 7 & 4 & 2 & 6 & 5 & 3 & 9 &
1\tabularnewline
\bottomrule
\end{longtable}

\begin{enumerate}
\def\labelenumi{\alph{enumi}.}
\item
  Find
\item
  Solve
\item
  Find
\item
  Solve
\end{enumerate}

10. Use the table below to fill in the missing values.

\begin{longtable}[]{@{}llllllllll@{}}
\toprule
$\textbf{t}} & 0 & 1 & 2 & 3 & 4 & 5 & 6 & 7 & 8\tabularnewline
\midrule
\endhead
$\textbf{h(t)}} & 6 & 0 & 1 & 7 & 2 & 3 & 5 & 4 & 9\tabularnewline
\bottomrule
\end{longtable}

\begin{enumerate}
\def\labelenumi{\alph{enumi}.}
\item
  Find
\item
  Solve
\item
  Find
\item
  Solve
\end{enumerate}

For each table below, create a table for

\begin{longtable}[]{@{}llll@{}}
\toprule
\begin{minipage}[t]{0.24\columnwidth}\raggedright\strut
11.\strut
\end{minipage} & \begin{minipage}[t]{0.24\columnwidth}\raggedright\strut
\begin{longtable}[]{@{}llllll@{}}
\toprule
$\textbf{x}} & 3 & 6 & 9 & 13 & 14\tabularnewline
\midrule
\endhead
$\textbf{f(x)}} & 1 & 4 & 7 & 12 & 16\tabularnewline
\bottomrule
\end{longtable}\strut
\end{minipage} & \begin{minipage}[t]{0.24\columnwidth}\raggedright\strut
\begin{quote}
12.
\end{quote}\strut
\end{minipage} & \begin{minipage}[t]{0.24\columnwidth}\raggedright\strut
\begin{longtable}[]{@{}llllll@{}}
\toprule
$\textbf{x}} & 3 & 5 & 7 & 13 & 15\tabularnewline
\midrule
\endhead
$\textbf{f(x)}} & 2 & 6 & 9 & 11 & 16\tabularnewline
\bottomrule
\end{longtable}\strut
\end{minipage}\tabularnewline
\bottomrule
\end{longtable}

For each function below, find ~

13. 14.

15. 16.

17. 18.

For each function, find a domain on which $f$ is one-to-one and
non-decreasing, then find the inverse of $f$ restricted to that
domain.

19. 20.

21. ~ 22.

23. If and , find

\begin{enumerate}
\def\labelenumi{\alph{enumi}.}
\item
\item
\item
  What does this tell us about the relationship between and ?
\end{enumerate}

24. If and , find

\begin{enumerate}
\def\labelenumi{\alph{enumi}.}
\item
\item
\item
  What does this tell us about the relationship between and ?
\end{enumerate}
