\section{Implicit Differentiation}
\label{sec:implicit}

Implicit Differentiation
In our work up until now, the functions we needed to differentiate were either given explicitly, such as y=x2+ex, or it was possible to get an explicit formula for them, such as solving y3-3x2=5 to get y=5+3x2------√3. Sometimes, however, we will have an equation relating x and y which is either difficult or impossible to solve explicitly for y, such as y+ey=x2. In any case, we can still find y′=f′(x) by using implicit differentiation.

The key idea behind implicit differentiation is to assume that y is a function of x even if we cannot explicitly solve for y. This assumption does not require any work, but we need to be very careful to treat y as a function when we differentiate and to use the Chain Rule.

\begin{example}
Assume that y is a function of x. Calculate

ddx(y3)
ddx(x3y2)
ddx(ln(y))
\begin{solution} We need the chain rule since y is a function of x:
ddx(y3)=3y2dydx=or3y2y′
We need to use the product rule and the Chain Rule:
ddx(x3y2)===orx3ddx(y2)+y2ddx(x3)x32ydydx+y23x22x3yy′+3y2x2
We know ddx(ln(x))=1x, so we use that and the Chain Rule:
ddx(ln(y))=1y⋅y′
\end{solution}\end{example}

Implicit Differentiation
To determine y′, differentiate each side of the defining equation, treating y as a funct\begin{example} x, and then algebraically solve for y′.

(The last example in the following video gets rather messy – don't worry too much if you can't follow all of the simplifications at the end.)

\begin{example}
Find the slope of the tangent line to the circle x2+y2=25 at the point (3,4) using implicit differentiation.

\begin{solution} We differentiate each side of the equation x2+y2=25 and then solve for y′:
ddx(x2+y2)=2x+2yy′=ddx(25)0
Solving for y′, we have y′=-2x2y=-xy, and, at the point (3,4),
y′=-34.
tangent to circle
\end{solution}\end{example}

In the previous example, it would have been easy to explicitly solve for y, and then we could differentiate y to get y′. Because we could explicitly solve for y, we had a choice of methods for calculating y′. Sometimes, however, we cannot explicitly solve for y, and the only way of determining y′ is to use implicit differentiation.

Related Rates
If several variables or quantities are related to each other and some of the variables are changing at a known rate, then we can use derivatives to determine how rapidly the other variables must be changing.

Here is a link to the examples used in the videos in this section: Related Rates.

\begin{example}
Suppose the border of a town is roughly circular, and the radius of that circle has been increasing at a rate of 0.1 miles each year. Find how fast the area of the town has been increasing when the radius is 5 miles.

picture
\begin{solution} We could get an approximate answer by calculating the area of the circle when the radius is 5 miles (A=πr2=π(5 miles)2\approx   78.6 miles2 ) and 1 year later when the radius is 0.1 feet larger than before (A=πr2=π(5.1 miles)2\approx   81.7 miles2 ) and then finding
ΔAreaΔtime=81.7 mi2-78.6 mi21 year=3.1 mi2/yr.
This approximate answer represents the average change in area during the 1 year period when the radius increased from 5 miles to 5.1 miles, and would correspond to the secant slope on the area graph.

To find the exact answer, though, we need derivatives. In this case both radius and area are functions of time:
r(t)= radius at time tA(t)= area at time t
We know how fast the radius is changing, which is a statement about the derivative: drdt=0.1mileyear. We also know that r=5 at our moment of interest.

We are looking for how fast the area is increasing, which is dAdt.

Now we need an equation relating our variables, which is the area equation:
A=πr2.
Taking the derivative of both sides of that equation with respect to t, we can use implicit differentiation:
ddt(A)=dAdt=ddt(πr2)π2rdrdt
Plugging in the values we know for r and drdt,
dAdt=π2(5 miles)(0.1milesyear)=πmiles2year
So the area of the town is increasing by approximately 3.14 square miles per year when the radius is 5 miles.
\end{solution}\end{example}
