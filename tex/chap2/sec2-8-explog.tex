\section{Exponential and Logarithmic Functions}
\label{sec:explogderiv}

\subsection{Derivatives of Exponential Functions}
Note the difference between power functions and exponential functions.
\begin{itemize}
    \item Power function: $f(x) = x^4$. $f'(x) = 4x^3$.
    \item Exponential function: $f(x) = 4^x$.
    \end{itemize}

To get an intuitive idea of the derivative of an exponential function, consider this graph of $y=e^x$. What can we say about the derivative of $e^x$? When $x$ is close to 0, the derivative is small. As $x$ increases, the derivative gets rapidly bigger. If we were to sketch $y=\frac{d}{dx}e^x$ on top of the graph of $y=e^x$, they would be very similar. In fact, they are the same!

\begin{figure}[ht!]
\centering
\begin{tikzpicture}[scale=1]
    % grid
    \draw[step=1, very thin, gray] (0, 0) grid (8, 6);

    % axes
    \draw[->] (-0.2,0) -- coordinate (x axis mid) (8.5,0) node[right] {$x$};
    \draw[->] (0,-0.2) -- coordinate (y axis mid) (0, 6.5) node[above] {$y$};

    % ticks
    \foreach \x in {1, 2, ..., 4}
     		\draw (2*\x,1pt) -- (2*\x,-3pt)
			node[anchor=north] {\x};
    	\foreach \y in {10, 20, ..., 60}
     		\draw (1pt,\y/10) -- (-3pt,\y/10) 
     			node[anchor=east] {\y}; 

    % labels
    %\node[below = 15] at (x axis mid) {Time (hours)};
	%\node[rotate=90, above = 25] at (y axis mid) {Bacteria population (millions)};

    % plot 
    \draw[smooth, samples=1000, domain=0:8, very thick, color=aldRed] plot(\x, {exp(\x/2)/10});    
    %\draw[samples=2, domain=0:10.2, very thick, color=aldBlue] plot(\x, {50*exp(-4) - 19*exp(-4)*(\x-8)});
    % dy/dx = 6e^(-x/2) - (3x+1)e^(-x/2) = (-3x+5)e^(-x/2)

    % points
    %\filldraw (0, 3.7) circle [radius=3pt]; 
    %\filldraw (8, 0.916) circle [radius=3pt]; 
    % legend

\end{tikzpicture}
\caption{$y=e^x$}
\label{fig:2-8-exp}
\end{figure} 

\begin{theorem}[Derivatives of Exponential Functions]
Let $a>0$ and let $k\in\R$.
\begin{itemize}
    \item $\displaystyle\dfrac{d}{dx}e^x=e^x$
    \item $\displaystyle\dfrac{d}{dx}a^x=\ln(a)a^x$
    \item $\displaystyle\dfrac{d}{dx}e^{kx} = ke^{kx}$
 \end{itemize}
\end{theorem}

{\bf Example: } What is the percentage rate of change of $f(x) = e^{kx}$?

{\bf Solution: } Percentage rate of change is 
$$\frac{f'(x)}{f(x)}\cdot 100\% = \frac{ke^{kx}}{e^{kx}}\cdot 100\% = k\cdot 100\% \enspace .$$
So $k$, as a percentage is the constant percentage rate of change of $f(x) = e^{kx}$. 

{\bf Example: } Suppose you invest \$1000 in a stock that grows at a constant rate of 15\% per year. At what rate is the value of your investment growing after 10 years?

{\bf Solution: } Let $A(t)$ be the value of the investment after $t$ years. Then $A(t) = Pe^{rt} = 1000e^{0.15t}$. $A'(t) = 1000\cdot 0.15e^{0.15t} = 150e^{0.15t}$ dollars per year. So $A'(10) = 150e^{0.15\cdot 10} = 150e^{1.5} = \$672.25$ per year.



\subsection{Derivatives of Logarithmic Functions}


To get an intuitive idea of the derivative of a logarithmic function, consider the graph of $y=\ln(x)$ in Figure \ref{fig:2-8-log}. What can we say about the derivative of $\ln(x)$? When $x$ is close to 0, the derivative is very large. As $x$ increases, $\ln(x)$ levels off, so the derivative goes to $0$. This is like the graph of $y=\frac{1}{x}$, which is in fact, the derivative of $\ln(x)$.

\begin{figure}[ht!]
\centering
\begin{tikzpicture}[scale=1]
    % grid
    \draw[step=1, very thin, gray] (0, -3) grid (15, 3);

    % axes
    \draw[->] (-0.2,0) -- coordinate (x axis mid) (15.5,0) node[right] {$x$};
    \draw[->] (0,-3.2) -- coordinate (y axis mid) (0, 3.5) node[above] {$y$};

    % ticks
    \foreach \x in {1, 2, ..., 15}
     		\draw (\x,1pt) -- (\x,-3pt)
			node[anchor=north] {\x};
    	\foreach \y in {-3, -2, ..., 3}
     		\draw (1pt,\y) -- (-3pt,\y) 
     			node[anchor=east] {\y}; 

    % labels
    %\node[below = 15] at (x axis mid) {Time (hours)};
	%\node[rotate=90, above = 25] at (y axis mid) {Bacteria population (millions)};

    % plot 
    \draw[smooth, samples=1000, domain=0.05:15, very thick, color=aldRed] plot(\x, {ln(\x)});  
    \draw[smooth, samples=1000, domain=0.32:15, very thick, color=aldBlue] plot(\x, {1/\x});  
    %\draw[samples=2, domain=0:10.2, very thick, color=aldBlue] plot(\x, {50*exp(-4) - 19*exp(-4)*(\x-8)});
    % dy/dx = 6e^(-x/2) - (3x+1)e^(-x/2) = (-3x+5)e^(-x/2)

    % points
    %\filldraw (0, 3.7) circle [radius=3pt]; 
    %\filldraw (8, 0.916) circle [radius=3pt]; 
    % legend

\end{tikzpicture}
\caption{$y=\ln(x)$ (Red) and $y=\frac{1}{x}$ (Blue)}
\label{fig:2-8-log}
\end{figure} 

\begin{theorem}[Derivatives of Logarithmic Functions]
    Let $b>0$, $b\neq 1$.
\begin{itemize}
    \item $\dfrac{d}{dx}\ln(x)=\dfrac{1}{x}$
    \item $\dfrac{d}{dx}\log_b(x)=\dfrac{1}{x\ln(b)}$
    \end{itemize}
\end{theorem}

\begin{example} Suppose the population of Napierville is $p(t) = 40.1 + 2.3\log(t)$ thousand people, $t$ years after 2000. Find and interpret $p'(30)$.

\begin{solution} $p'(t) = 0 + 2.3\dfrac{1}{t\ln(10)} \approx 0.9989\cdot \dfrac{1}{t}$ thousand people per year, so $p'(30) = 0.9989\cdot \dfrac{1}{30} = 0.333$ thousand people per year. This means that we would expect the population of Napierville to grow at a rate of 333 people per year in the year 2030.
\end{solution}\end{example}
