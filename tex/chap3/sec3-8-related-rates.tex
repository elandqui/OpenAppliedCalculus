\section{Related Rates}
\label{sec:related-rates}

Related Rates
When working with a related rates problem,

Draw a picture (if possible).
Identify the quantities that are changing, and assign them variables.
Find an equation that relates those quantities.
Differentiate both sides of that equation with respect to time.
Plug in any known values for the variables or rates of change.
Solve for the desired rate.
\begin{example}
A company has determined the demand curve for their product is q=5000-p2--------√, where p is the price in dollars, and q is the quantity in millions. If weather conditions are driving the price up \$2 a week, find the rate at which demand is changing when the price is \$40.

\begin{solution} The quantities changing are p and q, and we assume they are both functions of time, t, in weeks. We already have an equation relating the quantities, so we can implicitly differentiate it.
ddt(q)=dqdt=dqdt=ddt(5000-p2)1/212(5000-p2)-1/2ddt(5000-p2)12(5000-p2)-1/2(-2pdpdt)
Using the given information, we know the price is increasing by \$2 per week when the price is \$40, giving dpdt=2 when p=40. Plugging in these values,
dqdt=12(5000-402)-1/2(-2(40)(2))\approx   -1.37
Demand is falling by 1.37 million items per week.
\end{solution}\end{example}
