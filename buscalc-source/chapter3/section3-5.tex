\protect\hyperlink{main-nav}{≡} \protect\hyperlink{close-nav}{×}

\hypertarget{section-3.5-additional-integration-techniques}{%
\section{Section 3.5: Additional Integration
Techniques}\label{section-3.5-additional-integration-techniques}}

\hypertarget{integration-by-parts}{%
\subsection{Integration By Parts}\label{integration-by-parts}}

Integration by parts is an integration method which enables us to find
antiderivatives of some new functions such as \textbackslash{}(
\textbackslash{}ln(x) \textbackslash{}) as well as antiderivatives of
products of functions such as \textbackslash{}(
x\^{}2\textbackslash{}ln(x) \textbackslash{}) and \textbackslash{}(
xe\^{}x \textbackslash{}).

If the function we're trying to integrate can be written as a product of
two functions, \textbackslash{}(u\textbackslash{}), and
\textbackslash{}(dv\textbackslash{}), then integration by parts lets us
trade out a complicated integral for hopefully simpler one.

\hypertarget{integration-by-parts-formula}{%
\paragraph{Integration by Parts
Formula}\label{integration-by-parts-formula}}

\textbackslash{}{[} \textbackslash{}int u\textbackslash{}, dv =
uv-\textbackslash{}int v\textbackslash{}, du \textbackslash{}{]}

For definite integrals: \textbackslash{}{[} \textbackslash{}int\_a\^{}b
u\textbackslash{}, dv =
\textbackslash{}left.uv\textbackslash{}right{]}\_a\^{}b-\textbackslash{}int\_a\^{}b
v\textbackslash{}, du \textbackslash{}{]}

\hypertarget{example-1}{%
\paragraph{Example 1}\label{example-1}}

Integrate \textbackslash{}( \textbackslash{}int xe\^{}x\textbackslash{},
dx \textbackslash{}).

To use the Integration by Parts method, we break apart the product into
two parts: \textbackslash{}{[} u=x
\textbackslash{}qquad\textbackslash{}text\{and\}\textbackslash{}qquad
dv=e\^{}x\textbackslash{}, dx.\textbackslash{}{]}

We now calculate \textbackslash{}(du\textbackslash{}), the derivative of
\textbackslash{}(u\textbackslash{}), and
\textbackslash{}(v\textbackslash{}), the integral of
\textbackslash{}(dv\textbackslash{}): \textbackslash{}{[}
du=\textbackslash{}left(\textbackslash{}frac\{d\}\{dx\}
x\textbackslash{}right)\textbackslash{}, dx
\textbackslash{}qquad\textbackslash{}text\{and\}\textbackslash{}qquad v=
\textbackslash{}int e\^{}x\textbackslash{}, dx =
e\^{}x.\textbackslash{}{]}

Using the Integration by Parts formula, \textbackslash{}{[}
\textbackslash{}int xe\^{}x\textbackslash{}, dx=uv-\textbackslash{}int
v\textbackslash{}, du = xe\^{}x - \textbackslash{}int
e\^{}x\textbackslash{}, dx.\textbackslash{}{]}

Notice the remaining integral is simpler that the original, and one
which we can easily evaluate: \textbackslash{}{[} xe\^{}x -
\textbackslash{}int e\^{}x\textbackslash{}, dx = xe\^{}x-e\^{}x+C.
\textbackslash{}{]}

In the last example we could have chosen either
\textbackslash{}(x\textbackslash{}) or \textbackslash{}( e\^{}x
\textbackslash{}) as our \textbackslash{}(u\textbackslash{}), but had we
chosen \textbackslash{}( u=e\^{}x \textbackslash{}), the second integral
would have become messier, rather than simpler.

\hypertarget{rule-of-thumb}{%
\paragraph{Rule of Thumb}\label{rule-of-thumb}}

When selecting the \textbackslash{}(u\textbackslash{}) for Integration
by Parts, select a logarithmic expression if one is present. If not,
select an algebraic expression (like \textbackslash{}(x\textbackslash{})
or \textbackslash{}(dx\textbackslash{})).

(There is a larger decision tree that can be written down for choosing
\textbackslash{}( u \textbackslash{}) and \textbackslash{}( dv
\textbackslash{}), but since we're not looking at any trigonometric
functions in this course the rule above is sufficient for the functions
we're integrating.)

\hypertarget{example-2}{%
\paragraph{Example 2}\label{example-2}}

Integrate \textbackslash{}(
\textbackslash{}int\textbackslash{}limits\_1\^{}4\textbackslash{},
6x\^{}2\textbackslash{}ln(x)\textbackslash{}, dx \textbackslash{}).

Since this contains a logarithmic expression, we'll use it for our u:
\textbackslash{}{[} u=\textbackslash{}ln(x) \textbackslash{}qquad
\textbackslash{}text\{and\} \textbackslash{}qquad dv=
6x\^{}2\textbackslash{}, dx\textbackslash{}{]}

We now calculate \textbackslash{}(du\textbackslash{}) and
\textbackslash{}(v\textbackslash{}): \textbackslash{}{[}
du=\textbackslash{}frac\{1\}\{x\}dx \textbackslash{}qquad
\textbackslash{}text\{and\} \textbackslash{}qquad v= \textbackslash{}int
6x\^{}2\textbackslash{}, dx =
6\textbackslash{}frac\{x\^{}3\}\{3\}=2x\^{}3 \textbackslash{}{]}

Using the By Parts formula: \textbackslash{}{[}
\textbackslash{}int\_1\^{}4
6x\^{}2\textbackslash{}ln(x)\textbackslash{}, dx =
\textbackslash{}left.2x\^{}3\textbackslash{}ln(x)\textbackslash{}right{]}\_1\^{}4
- \textbackslash{}int\_1\^{}4
6x\^{}2\textbackslash{}frac\{1\}\{x\}\textbackslash{}, dx
\textbackslash{}{]}

We can simplify the expression in the integral on the right:
\textbackslash{}{[} \textbackslash{}int\_1\^{}4
6x\^{}2\textbackslash{}ln(x)\textbackslash{}, dx =
\textbackslash{}left.2x\^{}3\textbackslash{}ln(x)\textbackslash{}right{]}\_1\^{}4
- \textbackslash{}int\_1\^{}4 6x\textbackslash{}, dx \textbackslash{}{]}

The remaining integral is a basic one we can now evaluate:
\textbackslash{}{[} \textbackslash{}int\_1\^{}4
6x\^{}2\textbackslash{}ln(x)\textbackslash{}, dx =
\textbackslash{}left.2x\^{}3\textbackslash{}ln(x)\textbackslash{}right{]}\_1\^{}4
- \textbackslash{}left.3x\^{}2\textbackslash{}right{]}\_1\^{}4
\textbackslash{}{]}

Finally, we can evaluate the expressions: \textbackslash{}{[}
\textbackslash{}begin\{align*\} \textbackslash{}int\_1\^{}4
6x\^{}2\textbackslash{}ln(x)\textbackslash{}, dx=\&
\textbackslash{}left(\textbackslash{}left(2\textbackslash{}cdot
4\^{}3\textbackslash{}ln(4)\textbackslash{}right)-\textbackslash{}left(2\textbackslash{}cdot
1\^{}3\textbackslash{}ln(1)\textbackslash{}right)\textbackslash{}right)-\textbackslash{}left(\textbackslash{}left(3\textbackslash{}cdot
4\^{}2\textbackslash{}right)-\textbackslash{}left(3\textbackslash{}cdot
1\^{}2\textbackslash{}right)\textbackslash{}right)\textbackslash{}\textbackslash{}
=\& 128\textbackslash{}ln(4)-45\textbackslash{}\textbackslash{}
\textbackslash{}approx \& 132.446 \textbackslash{}end\{align*\}
\textbackslash{}{]}

\hypertarget{integration-using-tables-of-integrals}{%
\subsection{Integration Using Tables of
Integrals}\label{integration-using-tables-of-integrals}}

There are many techniques of integration we will not be studying. Many
of them lead to general formulas which can be compiled into a Table of
Integrals -- a type of cheat-sheet for integration.

For example, here are two entries you might find in a table of
integrals:

\hypertarget{table-of-integral-examples}{%
\paragraph{Table of Integral
Examples}\label{table-of-integral-examples}}

\textbackslash{}{[} \textbackslash{}begin\{align*\} \textbackslash{}int
\textbackslash{}frac\{1\}\{x\^{}2-a\^{}2\}\textbackslash{}, dx =\&
\textbackslash{}frac\{1\}\{2a\}\textbackslash{}ln\textbackslash{}left\textbar{}\textbackslash{}frac\{x-a\}\{x+a\}\textbackslash{}right\textbar{}+C
\textbackslash{}\textbackslash{} \textbackslash{}int
\textbackslash{}frac\{1\}\{\textbackslash{}sqrt\{x\^{}2+a\^{}2\}\}\textbackslash{},
dx =\&
\textbackslash{}ln\textbackslash{}left\textbar{}x+\textbackslash{}sqrt\{x\^{}2+a\^{}2\}\textbackslash{}right\textbar{}+C
\textbackslash{}end\{align*\} \textbackslash{}{]}

\hypertarget{example-3}{%
\paragraph{Example 3}\label{example-3}}

Integrate \textbackslash{}(
\textbackslash{}int\textbackslash{}frac\{5\}\{x\^{}2-9\}\textbackslash{},
dx \textbackslash{}).

This integral looks very similar to the form of the first integral in
the examples table. By employing the rule that allows us to pull out
constants, and by rewriting 9 as
\textbackslash{}(3\^{}2\textbackslash{}), we can better see the match.
\textbackslash{}{[}
\textbackslash{}int\textbackslash{}frac\{5\}\{x\^{}2-9\}\textbackslash{},
dx =
5\textbackslash{}int\textbackslash{}frac\{1\}\{x\^{}2-3\^{}2\}\textbackslash{},
dx \textbackslash{}{]}

Now we simply use the formula from the table, with \textbackslash{}(a =
3\textbackslash{}). \textbackslash{}{[} \textbackslash{}begin\{align*\}
\textbackslash{}int\textbackslash{}frac\{5\}\{x\^{}2-9\}\textbackslash{},
dx = \&
5\textbackslash{}int\textbackslash{}frac\{1\}\{x\^{}2-3\^{}2\}\textbackslash{},
dx \textbackslash{}\textbackslash{} =\&
5\textbackslash{}left(\textbackslash{}frac\{1\}\{2\textbackslash{}cdot
3\}\textbackslash{}ln\textbackslash{}left\textbar{}\textbackslash{}frac\{x-3\}\{x+3\}\textbackslash{}right\textbar{}\textbackslash{}right)+C
\textbackslash{}\textbackslash{} =\&
\textbackslash{}frac\{5\}\{6\}\textbackslash{}ln\textbackslash{}left\textbar{}\textbackslash{}frac\{x-3\}\{x+3\}\textbackslash{}right\textbar{}+C
\textbackslash{}end\{align*\} \textbackslash{}{]}

Sometimes we have to combine the table with other techniques we've
learned, like substitution.

\hypertarget{example-4}{%
\paragraph{Example 4}\label{example-4}}

Integrate \textbackslash{}(
\textbackslash{}int\textbackslash{}frac\{x\^{}2\}\{\textbackslash{}sqrt\{x\^{}6+16\}\}\textbackslash{},
dx \textbackslash{}).

This integral looks somewhat like the second integral in the example
table, but the power of x is incorrect, and there is an x2 in the
numerator which does not match. Trying to utilize this rule, we can try
to rewrite the denominator to look like
(something)\textbackslash{}(\^{}2\textbackslash{}). Luckily,
\textbackslash{}( x\^{}6 =
\textbackslash{}left(x\^{}3\textbackslash{}right)\^{}2
\textbackslash{}). \textbackslash{}{[}
\textbackslash{}int\textbackslash{}frac\{x\^{}2\}\{\textbackslash{}sqrt\{x\^{}6+16\}\}\textbackslash{},
dx =
\textbackslash{}int\textbackslash{}frac\{x\^{}2\}\{\textbackslash{}sqrt\{\textbackslash{}left(x\^{}3\textbackslash{}right)\^{}2+16\}\}\textbackslash{},
dx \textbackslash{}{]}

Now we can use substitution, letting \textbackslash{}( u=x\^{}3
\textbackslash{}), so \textbackslash{}( du=3x\^{}2\textbackslash{}, dx
\textbackslash{}).

Making the subsitution: \textbackslash{}{[}
\textbackslash{}int\textbackslash{}frac\{x\^{}2\}\{\textbackslash{}sqrt\{\textbackslash{}left(x\^{}3\textbackslash{}right)\^{}2+16\}\}\textbackslash{},
dx =
\textbackslash{}int\textbackslash{}frac\{1\}\{\textbackslash{}sqrt\{u\^{}2+16\}\}\textbackslash{},
\textbackslash{}frac\{du\}\{3\} =
\textbackslash{}frac\{1\}\{3\}\textbackslash{}int\textbackslash{}frac\{1\}\{\textbackslash{}sqrt\{u\^{}2+16\}\}\textbackslash{},
du \textbackslash{}{]}

Now we can use the table entry: \textbackslash{}{[}
\textbackslash{}frac\{1\}\{3\}\textbackslash{}int\textbackslash{}frac\{1\}\{\textbackslash{}sqrt\{u\^{}2+16\}\}\textbackslash{},
du =
\textbackslash{}frac\{1\}\{3\}\textbackslash{}ln\textbackslash{}left\textbar{}u+\textbackslash{}sqrt\{u\^{}2+16\}\textbackslash{}right\textbar{}+C
\textbackslash{}{]}

Undoing the substitution yields the final answer: \textbackslash{}{[}
\textbackslash{}int\textbackslash{}frac\{x\^{}2\}\{\textbackslash{}sqrt\{x\^{}6+16\}\}\textbackslash{},
dx =
\textbackslash{}frac\{1\}\{3\}\textbackslash{}ln\textbackslash{}left\textbar{}x\^{}3+\textbackslash{}sqrt\{x\^{}6+16\}\textbackslash{}right\textbar{}+C
\textbackslash{}{]}

\begin{longtable}[]{@{}ll@{}}
\toprule
\endhead
\href{section3-4.php}{← Previous Section} & \href{section3-6.php}{Next
Section →}\tabularnewline
\bottomrule
\end{longtable}
