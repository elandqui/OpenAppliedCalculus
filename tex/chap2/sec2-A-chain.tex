\section{The Chain Rule}
\label{sec:chain}

The big idea of the {\bf Chain Rule}\index{Chain rule} is that it allows us to find the derivative of a composition\index{Composition} of functions.

\begin{example}
Find the derivative of $y=(4x^3+15x)^2$.

\begin{solution} This is not a simple polynomial, so we can't use the basic building block rules yet. It is a product, so we could write it as $y=(4x^3+15x)^2=(4x^3+15x)(4x^3+15x)$ and use the Product Rule. Or we could multiply it out and simply differentiate the resulting polynomial. I'll do it the second way:
\begin{align*}
		y &= \left(4x^3+15x\right)^2\\
		 &= 16x^6+120x^4+225x^2\\
		y' &= 96x^5+480x^3+450x \enspace .
	\end{align*}
\end{solution}\end{example}

Now suppose we want to find the derivative of $y=\left(4x^3+15x\right)^{20}$. We could write it as a product with 20 factors and use the Product Rule, or we could multiply it out. But I don't want to do that. Do you? I didn't think so.

We need an easier way, a rule that will handle a composition like this. The Chain Rule is a little complicated, but it saves us the much more complicated algebra of multiplying something like this out. It will also handle compositions where it wouldn't be possible to multiply it out.

The Chain Rule is a common place for students to make mistakes. Part of the reason is that the notation takes a little getting used to. And part of the reason is that students often forget to use it when they should. When should you use the Chain Rule? Almost every time you take a derivative.

\begin{theorem}[Chain Rule]
Let $f(u)$ and $g(x)$ be differentiable functions, with $y=f(u)$ and $u=g(x)$.

{\bf Chain Rule (Leibniz notation)\index{Leibniz notation}}
$$\dfrac{dy}{dx}=\dfrac{dy}{du}\cdot\dfrac{du}{dx}$$
Notice that the $du$'s seem to cancel. This is one advantage of the Leibniz notation -- it can remind you of how the Chain Rule chains together.

{\bf Chain Rule (Lagrange notation)\index{Lagrange notation}}
$$f'(x)=f'(u)\cdot g'(x)=f'\left(g(x)\right)\cdot g'(x)$$

{\bf Chain Rule (in words)} \\
The derivative of a composition is the derivative of the outside (with the inside staying the same) times the derivative of the inside.
\end{theorem}
I recite the version in words each time I take a derivative, especially if the function is complicated.

\begin{example}
Find the derivative of $ y=\left(4x^3+15x\right)^2 $.

\begin{solution} This is the same one we did before by multiplying out. This time, let's use the Chain Rule: The inside function is what appears inside the parentheses: $4x^3+15x$. The outside function is the first thing we find as we come in from the outside -- it's the square function, $u^2$.

The derivative of this outside function is $2u$. Now using the Chain Rule, the derivative of our original function is $2u$ times the derivative of the inside, which is 
$$\dfrac{du}{dx} = \dfrac{d}{dx}(4x^3+15x) = 12x^2+15 \enspace .$$
So,
$$ y'=2\left(4x^3+15x\right)\left(12x^2+15 \right) \enspace .$$
\end{solution}\end{example}

If you multiply this out, you get the same answer we got before. Hurray! Algebra works!

\begin{example}
Find the derivative of $y=(4x^3+15x)^{20}$.

\begin{solution} Now we have a way to handle this one. It's the derivative of the outside times the derivative of the inside.

The outside function is $u^{20}$, which has derivative $u' = 20\cdot u^{19}$, so
$$y'=20\left(4x^3+15x\right)^{19}\left(12x^2+15\right) \enspace .$$
\end{solution}\end{example}

\begin{example}
Differentiate $y=e^{x^2+5}$.

\begin{solution} This isn't a simple exponential function; it's a composition. Typical calculator or computer syntax can help you see what the ``inside'' function is here. On a TI calculator, for example, when you push the $e^x$ key, it opens up parentheses: \verb|e^(|. This tells you that the ``inside'' of the exponential function is the exponent. Here, the inside is the exponent $u = x^2+5$. Now we can use the Chain Rule: We want the derivative of the outside times the derivative of the inside. The outside is $e^u$, so its derivative is $\dfrac{d}{du}e^u = e^u$. The derivative of the inside function is $\dfrac{d}{du}(x^2+5) = 2x$. So
$$\dfrac{d}{dx}\left( e^{x^2+5} \right)= \left( e^{x^2+5} \right)\cdot (2x).$$
\end{solution}\end{example}

\begin{example}
The table gives values for $f$ , $f'$ , $g$, and $g'$ at a number of points. Use these values to determine $( f \circ g )(x)$ and $( f \circ g ) '(x)$ at $x = -1$ and 0.
\begin{table}[ht!]
\centering
\begin{tabular}{ccccccc}
\toprule
$x$ & $f(x)$ & $g(x)$ &	$f'(x)$ & $g'(x)$ & $(f \circ g)(x)$ & $(g \circ f)(x)$ \\
\midrule
$-1$ & $2$	& $3$ &	$1$ & $0$ & & \\
$0$  & $-1$ & $1$ &	$3$ & $2$ & & \\
$1$  & $1$  & $0$ & $-1$ & $3$ & & \\
$2$	 & $3$ & $-1$ & $0$ & $1$ & & \\
$3$  & $0$ & $2$  &	$2$	& $-1$ & & \\
\bottomrule
\end{tabular}
\end{table}

\begin{solution} 
    \begin{itemize}[label={}]
    \item $(f\circ g)(-1) = f(g(-1))= f(3)=0$
    \item $(f\circ g)(0) = f(g(0))=f(1)=1$
    \item $(f\circ g)'(-1) = f'(g(-1))\cdot g'(-1)=f'(3)\cdot(0)=(2)(0)=0$ 
    \item $(f\circ g)'(0) = f'(g(0))\cdot g'(0)=f'(1)\cdot(2)=(-1)(2)=-2$
    \end{itemize}
\end{solution}\end{example}

\begin{example}
If 2400 people now have a disease, and the number of people with the disease appears to double every 3 years, then the number of people expected to have the disease in t years is $p(t)=2400\cdot 2^{t/3}$.
    \begin{enumerate}[label=(\alph*)]
    \item How many people are expected to have the disease in 2 years?

    \begin{solution}
    In 2 years, $p(2)=2400\cdot2^{2/3} \approx 3,810$ people will have the disease.
    \end{solution}
    \item When are 50,000 people expected to have the disease?

    \begin{solution}
    We know $p(t)=50{,}000$, for some $t$, so we need to solve $50{,}000=2400\cdot 2^{t/3}$ for $t$. We could start by isolating the exponential by dividing both sides by 2400:
    \begin{align*}
		\dfrac{50000}{2400} &= 2^{t/3} \\
		\ln\left(\dfrac{50000}{2400}\right) &= \ln\left(2^{t/3}\right) \qquad \text{(Taking the natural log of both sides.)}\\
		\ln\left(\dfrac{500}{24}\right) &= \dfrac{t}{3}\ln(2) \qquad \text{(Using the exponent property for logs.)}\\
		t &= \dfrac{3\ln\left(\dfrac{500}{24}\right)}{\ln(2)}\approx 13.14\text{ years}\qquad \text{(Solving for $ t $.)}
	\end{align*}
We expect 50,000 people to have the disease about 13.14 years from now.
    \end{solution}
    \item How fast is the number of people with the disease expected to grow now and 2 years from now?
    
    \begin{solution} 
This is asking for $p'(t)$ when $t= 0$ and 2 years. Using the Chain Rule,
    \begin{align*}
		\dfrac{dy}{dt} &= \dfrac{d}{dt}\left(2400\cdot 2^{t/3}\right) \\
		&= 2400\cdot 2^{t/3}\cdot \ln(2)\cdot\dfrac{1}{3} \\
		&\approx 554.5\cdot 2^{t/3}
	\end{align*}
So, at $t=0$, the rate of growth of the disease is approximately $554.5\cdot 2^0\approx 554.5$ people per year. In two years, the rate of growth will be approximately $554.5\cdot 2^{2/3}\approx 880$ people per year.
    \end{solution}
    \end{enumerate}
\end{example}

\subsection{Derivatives of Complicated Functions}
You're now ready to take the derivative of some mighty complicated functions. But how do you tell what rule applies first? Work your way in from the outside -- what do you encounter first? That's the first rule you need. Use the Product, Quotient, and Chain Rules to peel off the layers, one at a time, until you're all the way inside.

\begin{example}
Find $\dfrac{d}{dx}\left( e^{3x}\cdot\ln(5x+7) \right) $.

\begin{solution} Coming in from the outside, we see that this is a product of two (complicated) functions. So we'll need the Product Rule first. we'll fill in the pieces we know, and then we can figure the rest as separate steps and substitute in at the end:
$$\dfrac{d}{dx}\left( e^{3x}\cdot\ln(5x+7) \right)=\left( \dfrac{d}{dx}\left( e^{3x}\right)\right)\cdot\ln(5x+7)+ e^{3x}\cdot \left(\dfrac{d}{dx}\left(\ln(5x+7) \right)\right)$$
Now as separate steps, we'll find
$$\dfrac{d}{dx}\left( e^{3x}\right)=3e^{3x} \quad \text{ (using the Chain Rule)}$$ 
and 
$$\dfrac{d}{dx}\left(\ln(5x+7) \right)=\dfrac{1}{5x+7}\cdot 5 \quad \text{ (also using the Chain Rule)}.$$
Finally, to substitute these in their places:
$$\dfrac{d}{dx}\left( e^{3x}\cdot\ln(5x+7) \right)=\left( 3e^{3x}\right)\cdot\ln(5x+7)+ e^{3x}\cdot \left(\dfrac{1}{5x+7}\cdot 5\right)$$
(We can stop here -- we don't need to try to simplify any further.)
\end{solution}\end{example}

\begin{example}
Differentiate $ z=\left(\dfrac{3t^3}{e^t(t-1)}\right)^4 $.

\begin{solution} Don't panic! As we come in from the outside, what's the first thing we encounter? It's that fourth power. That tells us that this is a composition, a (complicated) function raised to the fourth power.

{\bf Step One: Use the Chain Rule.} The derivative of the outside times the derivative of the inside:
$$\dfrac{dz}{dt}=\dfrac{d}{dt}\left(\dfrac{3t^3}{e^t(t-1)}\right)^4=4\left(\dfrac{3t^3}{e^t(t-1)}\right)^3\cdot \dfrac{d}{dt}\left(\dfrac{3t^3}{e^t(t-1)}\right)$$
Now we're one step inside, and we can concentrate on just the $ \dfrac{d}{dt}\left(\dfrac{3t^3}{e^t(t-1)}\right) $ part. Now, as you come in from the outside, the first thing you encounter is a quotient -- this is the quotient of two (complicated) functions.

{\bf Step Two: Use the Quotient Rule.} The derivative of the numerator is straightforward, so we can just calculate it. The derivative of the denominator is a bit trickier, so we'll leave it for now:
$$ \dfrac{d}{dt}\left(\dfrac{3t^3}{e^t(t-1)}\right)=\dfrac{\left( 9t^2 \right)\left( e^t(t-1) \right)-\left( 3t^3 \right)\left( \dfrac{d}{dt}\left( e^t(t-1) \right) \right)}{\left(e^t(t-1)\right)^2} \enspace . $$
Now we've gone one more step inside, and we can concentrate on just the $ \dfrac{d}{dt}\left( e^t(t-1) \right) $ part, which involves a product.

{\bf Step Three: Use the Product Rule:}
$$ \dfrac{d}{dt}\left( e^t(t-1)\right) = \left( e^t \right)(t-1)+\left( e^t \right)(1)$$
And now we're all the way in -- no more derivatives to take!

{\bf Step Four:} Now it's just a question of substituting back -- be careful now!
$$ \dfrac{d}{dt}\left( e^t(t-1)\right) = \left( e^t \right)(t-1)+\left( e^t \right)(1) $$ 
so
$$ \dfrac{d}{dt}\left(\dfrac{3t^3}{e^t(t-1)}\right)=\dfrac{\left( 9t^2 \right)\left( e^t(t-1) \right)-\left( 3t^3 \right)\left( \left( e^t \right)(t-1)+\left( e^t \right)(1) \right)}{\left(e^t(t-1)\right)^2} $$ 
so
$$\dfrac{dz}{dt}=\dfrac{d}{dt}\left(\dfrac{3t^3}{e^t(t-1)}\right)^4=4\left(\dfrac{3t^3}{e^t(t-1)}\right)^3\cdot \left( \dfrac{\left( 9t^2 \right)\left( e^t(t-1) \right)-\left( 3t^3 \right)\left( \left( e^t \right)(t-1)+\left( e^t \right)(1) \right)}{\left(e^t(t-1)\right)^2} \right)$$
Phew!
\end{solution}\end{example}


