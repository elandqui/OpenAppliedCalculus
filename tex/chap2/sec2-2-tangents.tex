\section{Tangent Lines and Instantaneous Rates of Change}
\label{sec:tangents}

Can't-quite-do-it-yet Definition (Tangent Line)
A tangent line is a line at one point on a curve… that does its best to be the curve at that point?

As you may be able to see in the image below, the closer the point Q is to the point P, the closer the secant slope gets to the tangent slope. This will be key to finding the tangent slope, but first we need to more carefully define the idea of getting closer to.

secant to tangent

Instantaneous Velocity
Suppose we drop a tomato from the top of a 100 foot building and time its fall.

graph and table
Some questions are easy to answer directly from the table:

How long did it take for the tomato to drop 100 feet? (2.5 seconds)
How far did the tomato fall during the first second? (100 – 84 = 16 feet)
How far did the tomato fall during the last second? (64 – 0 = 64 feet)
How far did the tomato fall between t=0.5 and t=1? (96 – 84 = 12 feet)
Some questions require a little calculation:

What was the average velocity of the tomato during its fall?
Average velocity=distance fallentotal time=ΔpositionΔtime=-100 ft2.5 s=-40 ft/s
What was the average velocity between t=1 and t=2 seconds?
Average velocity=ΔpositionΔtime=36 ft-84 ft2 s-1 s=-48 ft1 s=-48 ft/s
Some questions are more difficult:

How fast was the tomato falling 1 second after it was dropped?

This question is significantly different from the previous two questions about average velocity. Here we want the instantaneous velocity, the velocity at an instant in time. Unfortunately the tomato is not equipped with a speedometer so we will have to give an approximate answer.

One crude approximation of the instantaneous velocity after 1 second is simply the average velocity during the entire fall, -40 ft/s . But the tomato fell slowly at the beginning and rapidly near the end so the "-40 ft/s" estimate may or may not be a good answer.

We can get a better approximation of the instantaneous velocity at t=1 by calculating the average velocities over a short time interval near t=1. The average velocity between t=0.5 and t=1 is -12 feet0.5 s=-24 ft/s, and the average velocity between t=1 and t=1.5 is -20 feet0.5 s=-40 ft/s so we can be reasonably sure that the instantaneous velocity is between -24 ft/s and -40 ft/s.

In general, the shorter the time interval over which we calculate the average velocity, the better the average velocity will approximate the instantaneous velocity. The average velocity over a time interval is ΔpositionΔtime, which is the slope of the secant line through two points on the graph of height versus time. The instantaneous velocity at a particular time and height is the slope of the tangent line to the graph at the point given by that time and height.

graph
Average vs Instantaneous Velocity
Average velocity = ΔpositionΔtime = slope of the secant line through 2 points.

Instantaneous velocity = slope of the line tangent to the graph.

Growing Bacteria
Suppose we set up a machine to count the number of bacteria growing on a Petri plate. At first there are few bacteria so the population grows slowly. Then there are more bacteria to divide so the population grows more quickly. Later, there are more bacteria and less room and nutrients available for the expanding population, so the population grows slowly again. Finally, the bacteria have used up most of the nutrients, and the population declines as bacteria die.

 bacteria 1
The population graph can be used to answer a number of questions.

What is the bacteria population at time t=3 days?

From the graph, at t=3, the population is about 0.5 thousand, or 500 bacteria.

What is the population increment from t=3 to t=10 days?

At t=10, the population is about 4.5 thousand, so the increment is about 4000 bacteria.

What is the rate of population growth from t=3 to t=10 days?

The rate of growth from t=3 to t=10 is the average change in population during that time:
average change in population ===\approx   change in populationchange in timeΔpopulationΔtime4000 bacteria7 days570 bacteria/day.
This is the slope of the secant line through the two points (3, 500) and (10, 4500).

What is the rate of population growth on the third day, at t=3 ?

This question is asking for the instantaneous rate of population change, the slope of the line which is tangent to the population curve at (3, 500). If we sketch a line approximately tangent to the curve at (3, 500) and pick two points near the ends of the tangent line segment , we can estimate that instantaneous rate of population growth is approximately 320 bacteria/day .

 bacteria 2
Tangent Lines
Try this!
The graph below is the graph of y=f(x). We want to find the slope of the tangent line at the point (1, 2).

graph
First, draw the secant line between (1, 2) and (2, -1) and compute its slope.

Now draw the secant line between (1, 2) and (1.5, 1) and compute its slope.

Compare the two lines you have drawn. Which would be a better approximation of the tangent line to the curve at (1, 2)?

Now draw the secant line between (1, 2) and (1.3, 1.5) and compute its slope. Is this line an even better approximation of the tangent line?

Now draw your best guess for the tangent line and measure its slope. Do you see a pattern in the slopes?

You should have noticed that as the interval got smaller and smaller, the secant line got closer to the tangent line and its slope got closer to the slope of the tangent line. That’s good news–we know how to find the slope of a secant line.

In some applications, we need to know where the graph of a function f(x) has horizontal tangent lines (slopes = 0).

\begin{example}
Below is the graph of y=g(x). At what values of x does the graph of g(x) have horizontal tangent lines?

graph
\begin{solution}
The tangent lines to the graph of g(x) are horizontal (slope = 0) when x\approx   -1,1,2.5, and 5.
\end{solution}\end{example}

Let's explore further this idea of finding the tangent slope based on the secant slope.

\begin{example}
Find the slope of the line $L$ in the graph below which is tangent to $f(x)=x^2$ at the point $(2,4)$.

\begin{solution} We could estimate the slope of L from the graph, but we won't. Instead, we will use the idea that secant lines over tiny intervals approximate the tangent line.

graphgraph
We can see that the line through (2,4) and (3,9) on the graph of f is an approximation of the slope of the tangent line, and we can calculate that slope exactly: m=ΔyΔx=9-43-2=5. But m=5 is only an estimate of the slope of the tangent line and not a very good estimate. It's too big. We can get a better estimate by picking a second point on the graph of f which is closer to (2,4) – the point (2,4) is fixed and it must be one of the points we use.

From the second figure, we can see that the slope of the line through the points (2,4) and (2.5,6.25) is a better approximation of the slope of the tangent line at (2,4): m=ΔyΔx=6.25-42.5-2=2.250.5=4.5, a better estimate, but still an approximation. We can continue picking points closer and closer to (2,4) on the graph of f, and then calculating the slopes of the lines through each of these points and the point (2,4):

Points to the left of (2,4)
x	y=x2	Slope of line through (x,y) and (2,4).	1.5	2.25	3.5	1.9	3.61	3.9	1.99	3.9601	3.99
Points to the right of (2,4)
x	y=x2	Slope of line through (x,y) and (2,4).	3	9	5	2.5	6.25	4.5	2.01	4.0401	4.01
The only thing special about the x–values we picked is that they are numbers which are close, and very close, to x=2. Someone else might have picked other nearby values for x. As the points we pick get closer and closer to the point (2,4) on the graph of y=x2, the slopes of the lines through the points and (2,4) are better approximations of the slope of the tangent line, and these slopes are getting closer and closer to 4.

We can bypass much of the calculating by not picking the points one at a time: let's look at a general point near (2,4). Define x=2+h so h is the increment from 2 to x. If h is small, then x=2+h is close to 2 and the point (2+h,f(2+h))=(2+h,(2+h)2) is close to (2,4). The slope m of the line through the points (2,4) and (2+h,(2+h)2) is a good approximation of the slope of the tangent line at the point (2,4):

graph
m=(2+h)2-4(2+h)-2=(4+4h+h2)-4h=4h+h2h=h(4+h)h=4+h
The value m=4+h is the slope of the secant line through the two points (2,4) and (2+h,(2+h)2). As h gets smaller and smaller, this slope approaches the slope of the tangent line to the graph of f at (2,4).

More formally, we could write:
Slope of the tangent line=ΔyΔx=limh→0(4+h).
We can easily evaluate this limit using direct substitution, finding that as the interval h shrinks towards 0, the secant slope approaches the tangent slope, 4.
\end{solution}\end{example}

The tangent line problem and the instantaneous velocity problem are the same problem. In each problem we wanted to know how rapidly something was changing at an instant in time, and the answer turned out to be finding the slope of a tangent line, which we approximated with the slope of a secant line. This idea is the key to defining the slope of a curve.
