\section{Sum, Difference, and Constant Multiple Rules}
\label{sec:sumdiff}

In the next few sections, we'll get the derivative rules that will let us find formulas for derivatives when our function comes to us as a formula.

\subsection{Sum and Difference}
Given two functions, $f(x)$ and $g(x)$, what can be said about the derivative of their sum? Think about it. What should we expect for the slope of the curve $y = f(x)+g(x)$ or the curve $y=f(x) - g(x)$? We can prove the following result using the limit definition of the derivative, but it should make intuitive sense that the slope of a sum is the sum of the slopes.

\begin{theorem}[Sum and Difference Rule\index{Sum and difference rule}\index{Derivative rules!Sum and difference rule}]
\label{thm:sumdiffderiv}
If $f(x)$ and $g(x)$ are differentiable, then
    $$\dfrac{d}{dx}(f(x) \pm g(x))=f'(x) \pm g'(x) \enspace .$$
\end{theorem}

\subsection{Constant Multiples}
Given a single function, $f(x)$, we have $2f(x) = f(x) + f(x)$ and $3f(x) = f(x) + f(x) + f(x)$, so it would make sense that the slope of $y=k\cdot f(x)$ is $k$ times the slope of $y=f(x)$. This too, can be proved using the limit definition of the derivative. We often say that when taking the derivative of a multiple of a function, that ``constants come along for the ride.''

\begin{theorem}[Constant Multiple Rule\index{Constant multiple rule}\index{Derivative rules!Constant multiple rule}]
\label{thm:constmultderiv}
If $f(x)$ is differentiable and $k$ is a real number, then
$$\dfrac{d}{dx}(k\cdot f(x))=k\cdot f'(x) \enspace .$$
\end{theorem}

\begin{example}
    Suppose $f'(2) = 8$ and $g'(2) = 9$.
    \begin{enumerate}
    \item If $h(x) = f(x) + g(x)$, $h'(2) = f'(2) + g'(2) = 8+9 = 17$.
    \item If $h(x) = f(x) - g(x)$, $h'(2) = f'(2) - g'(2) = 8-9 = -1$.
    \item If $h(x) = 3\cdot f(x) + 2\cdot g(x)$, $h'(2)= 3\cdot f'(2) + 2\cdot g'(2) = 3\cdot 8 + 2\cdot 9 =24+18 = 42$.
    \end{enumerate}
\end{example}

\subsection{Exercises}
\begin{enumerate}
    \item Prove Theorem \ref{thm:sumdiffderiv} using the limit definition of the derivative, Definition \ref{def:deriv}.
    \item Prove Theorem \ref{thm:constmultderiv} using the limit definition of the derivative, Definition \ref{def:deriv}.
    \item Suppose $f'(3) = 4$ and $g'(3) = 2$.
    \begin{enumerate}
    \item If $h(x) = f(x) + g(x)$, find $h'(3)$.
    \item If $h(x) = f(x) - g(x)$, find $h'(3)$.
    \item If $h(x) = 5\cdot f(x) - 4\cdot g(x)$, find $h'(3)$.
    \end{enumerate}
\end{enumerate}
