\section{Accumulation}
\label{sec:accumulation}
% From Section 3.1

\subsection{Accumulation in Real Life}
We have already seen that the area under a graph can represent quantities whose units are not the usual geometric units of square meters or square feet. For example, if $t$ is a measure of time in seconds and $f(t)$ is a velocity with units of feet/second, then the definite integral has units (feet/second)$\cdot$(seconds) $=$ feet.

In general, the units for the definite integral $\int_a^bf(x)\,dx$ are ($y$-units)$\cdot$($x$-units). A quick check of the units can help avoid errors in setting up an applied problem.

In previous examples, we looked at a function that represented a rate of travel (miles per hour); in that case, the area represented the total distance traveled. For functions representing other rates such as the production of a factory (bicycles per day), or the flow of water in a river (gallons per minute) or traffic over a bridge (cars per minute), or the spread of a disease (newly sick people per week), the area will still represent the total amount of something.

\begin{example}
Suppose $MR(q)$ is the marginal revenue in dollars/item for selling $q$ items. What does $\int_0^{150}MR(q)\,dq$ represent?

\begin{solution}
  $\int_0^{150}MR(q)\,dq$ has units (dollars/item)$\cdot$(items) $=$ dollars, and represents the accumulated dollars for selling from 0 to 150 items. That is, $\int_0^{150}MR(q)\,dq=TR(150)$, the total revenue from selling 150 items.
\end{solution}\end{example}

\begin{example}
Suppose $r(t)$ represents the rate of change of the diameter of a tree, in centimeters per year. What does $\int_{T_1}^{T_2}r(t)\,dt$ represent?

\begin{solution}
  $\int_{T_1}^{T_2}r(t)\,dt$ has units of (centimeters per year)$\cdot$(years) $=$ centimeters, and represents the accumulated growth of the tree's diameter from time $T_1$ years to $T_2$ years. That is, $\int_{T_1}^{T_2}r(t)\,dt$ is the change in the diameter of the tree over this period of time.
\end{solution}\end{example}
