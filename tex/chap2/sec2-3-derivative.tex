\section{The Derivative}
\label{sec:derivative}

The Derivative
We can view the derivative in different ways. Here are a three of them:

The derivative of a function f at a point (x, f(x)) is the instantaneous rate of change.
The derivative is the slope of the tangent line to the graph of f at the point (x,f(x)).
The derivative is the slope of the curve f(x) at the point (x,f(x)).
A function is called differentiable at (x,f(x)) if its derivative exists at (x,f(x)).

Notation for the Derivative
The derivative of y=f(x) with respect to x is written as
f′(x)
(read aloud as "f prime of x"), or
y′
(read aloud as "why prime") or
dydx
(read aloud as "dee why dee ex"), or
dfdx.
The notation that resembles a fraction is called Leibniz notation. It displays not only the name of the function (f or y), but also the name of the variable (in this case, x). It looks like a fraction because the derivative is a slope. In fact, this is simply ΔyΔx written in Roman letters instead of Greek letters.

Verb Forms
We find the derivative of a function, or take the derivative of a function, or differentiate a function.

We use an adaptation of the dfdx notation to mean "find the derivative of f(x):"
ddx[f(x)]=dfdx.
[The book uses parentheses instead of brackets–both are acceptable forms of the notation.]


Looking Ahead
We will have methods for computing exact values of derivatives from formulas soon. If the function is given to you as a table or graph, you will still need to approximate this way.

This is the foundation for the rest of this chapter. It’s remarkable that such a simple idea (the slope of a tangent line) and such a simple definition (for the derivative f′(x)) will lead to so many important ideas and applications.


The Derivative as a Function
We now know how to find (or at least approximate) the derivative of a function for any x-value; this means we can think of the derivative as a function, too. The inputs are the same x’s; the output is the value of the derivative at that x value.

\begin{example}
Below is the graph of a function y=f(x). We can use the information in the graph to fill in a table showing values of f′(x):
graph

\begin{solution} At various values of x, draw your best guess at the tangent line and measure its slope. You might have to extend your lines so you can read some points. In general, your estimate of the slope will be better if you choose points that are easy to read and far away from each other. Here are estimates for a few values of x (parts of the tangent lines used are shown above in the graph):

x	y=f(x)	f′(x)= the estimated slope of the tangent line to the curve at the point (x,y).	0	0	1	1	1	0	2	0	-1	3	-1	0	3.5	0	2
We can estimate the values of f′(x) at some non-integer values of x, too: f′(0.5)\approx   0.5 and f′(1.3)\approx   -0.3.

We can even think about entire intervals. For example, if 0<x<1, then f(x) is increasing, all the slopes are positive, and so f′(x) is positive.

The values of f′(x) definitely depend on the values of x, and f′(x) is a function of x. We can use the results in the table to help sketch the graph of f′(x).

derivative graph
\end{solution}\end{example}
\begin{example}
Shown is the graph of the height h(t) of a rocket at time t.

graph
Sketch the graph of the velocity of the rocket at time t. (Velocity is the derivative of the height function, so it is the slope of the tangent to the graph of position or height.)

\begin{solution} We can estimate the slope of the function at several points. The lower graph below shows the velocity of the rocket. This is v(t)=h′(t).

derivative
\end{solution}\end{example}
We can also find derivative functions algebraically using limits.



Interpreting the Derivative
So far we have emphasized the derivative as the slope of the line tangent to a graph. That interpretation is very visual and useful when examining the graph of a function, and we will continue to use it. Derivatives, however, are used in a wide variety of fields and applications, and some of these fields use other interpretations. The following are a few interpretations of the derivative that are commonly used.

General
Rate of Change: f′(x) is the rate of change of the function at x. If the units for x are years and the units for f(x) are people, then the units for dfdx are peopleyear, a rate of change in population.

Graphical
Slope: f′(x) is the slope of the line tangent to the graph of f at the point (x,f(x)).

Physical
Velocity: If f(x) is the position of an object at time x, then f′(x) is the velocity of the object at time x. If the units for x are hours and f(x) is distance measured in miles, then the units for f′(x)=dfdx are mileshour, miles per hour, which is a measure of velocity.

Acceleration: If f(x) is the velocity of an object at time x, then f′(x) is the acceleration of the object at time x. If the units are for x are hours and f(x) has the units mileshour, then the units for the acceleration f′(x)=dfdx are miles/hourhour=mileshour2, miles per hour per hour.

Business
Marginal Cost, Marginal Revenue, and Marginal Profit: We'll explore these terms in more depth later in the section. Basically, the marginal cost is approximately the additional cost of making one more object once we have already made x objects. If the units for x are bicycles and the units for f(x) are dollars, then the units for f′(x)=dfdx are dollars bicycle, the cost per bicycle.

In business contexts, the word "marginal" usually means the derivative or rate of change of some quantity.

One of the strengths of calculus is that it provides a unity and economy of ideas among diverse applications. The vocabulary and problems may be different, but the ideas and even the notations of calculus are still useful.

\begin{example}
Suppose the demand curve for widgets was given by D(p)=1p, where D is the quantity of widgets, in thousands, at a price of p dollars. Interpret the derivative of D at p=\$3.

\begin{solution} Note that we calculated D′(3) earlier to be D′(3)=-19\approx   -0.111.

Since D has units thousands of widgets and the units for p is dollars of price, the units for D′ will be thousands of widgetsdollar of price. In other words, it shows how the demand will change as the price increases.

Specifically, D′(3)\approx   -0.111 tells us that when the price is \$3, the demand will decrease by about 0.111 thousand items for every dollar the price increases.
\end{solution}\end{example} 
